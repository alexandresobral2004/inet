\documentclass{book}
\usepackage{a4wide}

%% possible fonts -- in order of preference
%%\usepackage{palatino}
\usepackage{bookman}
%%\usepackage{charter}
%%\usepackage{newcent}
%%\usepackage{times}
%%\usepackage{avant}
%%\usepackage{helvet}
%%\usepackage{sans}
%%\usepackage{chancery}

\usepackage[svgnames]{xcolor}	% for color text support
\usepackage[T1]{fontenc}
\usepackage[11pt]{moresize}
\usepackage{setspace}
\usepackage{ifpdf}
\usepackage{makeidx}
\usepackage{longtable}  %% page wrapping table environment
\usepackage{colortbl}   %% colors for tables
\usepackage{fancyvrb}   %% the "Verbatim" environment
\usepackage{fancyhdr}   %% custom headers and footers
\usepackage{multicol}
\usepackage{listings}   %% source code listings with syntax highlight (lstxxx commands)
\usepackage[tight]{shorttoc}   %% for generating a second table of contents, only containing chapter titles
\usepackage{bytefield}  %% for drawing protocol frames
\usepackage{paralist}   %% for compact lists
\usepackage[nottoc]{tocbibind}  %% makes Bibliography and Index show up in TOC
\settocbibname{References}

\setlength{\textwidth}{160mm}
%\setlength{\oddsidemargin}{12.5mm}
%\setlength{\evensidemargin}{12.5mm}
%\setlength{\topmargin}{0mm}
\setlength{\textheight}{220mm}
%\setlength{\parskip}{1ex}
%\setlength{\parindent}{5ex}

\renewcommand{\bottomfraction}{0.9}
\renewcommand{\topfraction}{0.9}
\renewcommand{\floatpagefraction}{0.9}

%% try to cure overfull hboxes
%% \tolerance=500

%% for navigation in dvi files, only needed by old teTeX versions
%%\usepackage{srcltx}

%% try this for spell checking: cat ess2002.tex | ispell -l -t -a | sort | uniq | more

%%
%% The following snippet changes the horizontal spacing between the number and
%% the title in the table of contents.
%%
%% http://tex.stackexchange.com/questions/33841/how-to-modify-the-space-between-the-numbers-and-text-of-sectioning-titles-in-the
%%
\makeatletter
 \renewcommand*\l@section{\@dottedtocline{1}{2em}{3em}}
 \renewcommand*\l@subsection{\@dottedtocline{2}{5em}{4em}}
\renewcommand*\l@chapter[2]{%
  \ifnum \c@tocdepth >\m@ne
    \addpenalty{-\@highpenalty}%
    \vskip 1.0em \@plus\p@
    \setlength\@tempdima{2em}%
    \begingroup
      \parindent \z@ \rightskip \@pnumwidth
      \parfillskip -\@pnumwidth
      \leavevmode \bfseries
      \advance\leftskip\@tempdima
      \hskip -\leftskip
      #1\nobreak\hfil \nobreak\hb@xt@\@pnumwidth{\hss #2}\par
      \penalty\@highpenalty
    \endgroup
  \fi}
\makeatother

%%
%% OMNeT++ logo, use as {\opp}
%%
\makeatletter
%%\DeclareRobustCommand{\omnetpp}{OM\-NeT\kern-.18em++\@}
\DeclareRobustCommand{\omnetpp}{OMNeT++\@}
\makeatother

\newcommand{\opp}{\omnetpp}

%%
%% PDF Header
%%
% note: \ifpdf now comes from the ifpdf package
%\newif\ifpdf
%\ifx\pdfoutput\undefined
%  \pdffalse
%\else
%  \pdfoutput=1
%  \pdftrue
%\fi
%% PDF-Info
\ifpdf
  \usepackage[pdftex]{graphicx}
  \usepackage[plainpages=false,linktocpage,bookmarksnumbered=true,pdftex]{hyperref}   %% automatic hyperlinking
  \pdfcompresslevel=9
  \pdfinfo{/Author (Andras Varga and others)
    /Title (INET Framework Developer's Guide)
    /Subject ()
    /Keywords (INET, INETMANET, OMNeT++, manual)}
\else
  \usepackage{graphicx}
  \usepackage[plainpages=false]{hyperref}   %% automatic hyperlinking
\fi

%%
%% Draft conditional to include unfinished parts
%%
\newif\ifdraft
\draftfalse %% uncomment for final version
%\drafttrue %% uncomment for draft version

%%
%% Generate Index
%%
\makeindex


%%
%% Link colors (hyperref package)
%%
\definecolor{MyDarkBlue}{rgb}{0.16,0.16,0.5}
%% XXX the next line apparently screws up all links except in TOC! they'll be colored nicely, but won't work.
%\hypersetup{
%    colorlinks=true,
%    linkcolor=MyDarkBlue,
%    anchorcolor=MyDarkBlue,
%    citecolor=MyDarkBlue,
%    filecolor=MyDarkBlue,
%    menucolor=MyDarkBlue,
%    runcolor=MyDarkBlue,
%    urlcolor=blue,
%}

%%
%% Heading and Footer
%%
\pagestyle{fancy}
\fancyhf{}
\renewcommand{\footrulewidth}{0.5pt}
\renewcommand{\chaptermark}[1]{\markboth{#1}{}}
\lhead{INET Framework Developer's Guide -- \leftmark}
\rfoot{\thepage}

%% this is used for chapter start pages
\fancypagestyle{plain}{
    \rfoot{\thepage}
}

%%
%% Use \begin{graybox}...\end{graybox} for notes
%%
\definecolor{MyGray}{rgb}{0.85,0.85,1.0}
\makeatletter\newenvironment{graybox}%
   {\begin{flushright}\begin{lrbox}{\@tempboxa}\begin{minipage}[r]{0.95\textwidth}}%
   {\end{minipage}\end{lrbox}\colorbox{MyGray}{\usebox{\@tempboxa}}\end{flushright}}%
\makeatother


\newenvironment{note}{\begin{graybox}\textbf{NOTE: }}{\end{graybox}}
\newenvironment{warning}{\begin{graybox}\textbf{WARNING: }}{\end{graybox}}
\newenvironment{caution}{\begin{graybox}\textbf{CAUTION: }}{\end{graybox}}
\newenvironment{rationale}{\begin{graybox}\textbf{Rationale: }}{\end{graybox}}
\newenvironment{important}{\begin{graybox}\textbf{IMPORTANT: }}{\end{graybox}}

%%
%% Set up listings package
%%
\lstloadlanguages{C++,make,perl,tcl,XML,R,Matlab}

%% See listings.pdf,pp20
\lstdefinelanguage{NED} {
    morekeywords={allowunconnected,bool,channel,channelinterface,connections,const,
                  default,double,extends,false,for,gates,if,import,index,inout,input,
                  int,like,module,moduleinterface,network,output,package,parameters,
                  property,simple,sizeof,string,submodules,this,true,types,volatile,
                  xml,xmldoc},
    sensitive=true,
    morecomment=[l]{//},
    morestring=[b]",
}
\lstdefinelanguage{MSG} {
    morekeywords={abstract,bool,char,class,cplusplus,double,enum,extends,false,
                  fields,int,long,message,namespace,noncobject,packet,properties,
                  readonly,short,string,struct,true,unsigned},
    sensitive=true,
    morecomment=[l]{//},
    morestring=[b]",
}
\lstdefinelanguage{inifile} {
    morekeywords={},
    sensitive=true,
    morecomment=[l]{\#},
    morestring=[b]",
}
\lstdefinelanguage{pseudocode} {
    morekeywords={if,then,else,otherwise,whenever,while},
    sensitive=true,
    morecomment=[l]{//},
    morestring=[b]",
    mathescape=true,
}

%% thick ruler on the left; also, designate backtick as LaTeX escape character
%% (e.g. \opp needs to be written as `\opp` inside listing blocks)
\lstset{
    escapechar=`,
    basicstyle=\ttfamily,
    identifierstyle=\color{Black},
    stringstyle=\color{DarkBlue},
    commentstyle=\color{SeaGreen},
    keywordstyle=\bfseries\color{Purple},
    showstringspaces=false,
    frame=leftline,
    framesep=10pt,
    framerule=3pt,
    xleftmargin=15pt
}

\definecolor{NEDRulerColor}{rgb}{0.5,1.0,0.5}  % pale green
\definecolor{MSGRulerColor}{rgb}{0.5,1.0,0.5}  % pale green
\definecolor{CPPRulerColor}{rgb}{0.8,0.5,0.2}  % pale orange
\definecolor{IniRulerColor}{rgb}{0.9,0.9,0.3}  % pale yellow
\definecolor{FileListingRulerColor}{rgb}{0.85,0.85,0.85}  % grey
%\definecolor{CommandLineRulerColor}{rgb}{0.9,0.9,0.2}
\definecolor{PseudoCodeRulerColor}{rgb}{0.0,1.0,1.0}  % cyan
\definecolor{XMLRulerColor}{rgb}{0.8,0.8,1.0}  % pale blue

%% See listings.pdf,pp39
\lstnewenvironment{ned}
    {\lstset{language=NED,rulecolor=\color{NEDRulerColor}}}
    {}
\lstnewenvironment{msg}
    {\lstset{language=MSG,rulecolor=\color{MSGRulerColor}}}
    {}
\lstnewenvironment{cpp}
    {\lstset{language=C++,rulecolor=\color{CPPRulerColor}}}
    {}
\lstnewenvironment{inifile}
    {\lstset{language=inifile,rulecolor=\color{IniRulerColor}}}
    {}
\lstnewenvironment{filelisting}
    {\lstset{language={},rulecolor=\color{FileListingRulerColor}}}
    {}
\lstnewenvironment{commandline}
    {\lstset{language={},framesep=11pt,framerule=1pt,xleftmargin=16pt}}
    {}
\lstnewenvironment{pseudocode}
    {\lstset{language=pseudocode,rulecolor=\color{PseudoCodeRulerColor}}}
    {}
\lstnewenvironment{XML}
    {\lstset{language=XML,rulecolor=\color{XMLRulerColor}}}
    {}

% add caption={#2} to display caption
\newcommand{\xmlsnippet}[2]{%
    \lstinputlisting[language=XML,rulecolor=\color{XMLRulerColor},linerange=<!\-\-#1\-\->-<!\-\-End\-\->,includerangemarker=false,firstnumber=0]{lib/Snippets.xml}}
\newcommand{\cppsnippet}[2]{%
    \lstinputlisting[language=C++,rulecolor=\color{CPPRulerColor},linerange=//!#1-//!End,includerangemarker=false,firstnumber=0]{lib/Snippets.cc}}
\newcommand{\msgsnippet}[2]{%
    \lstinputlisting[language=msg,rulecolor=\color{MSGRulerColor},linerange=//!#1-//!End,includerangemarker=false,firstnumber=0]{lib/Snippets.msg}}
\newcommand{\nedsnippet}[2]{%
    \lstinputlisting[language=ned,rulecolor=\color{NEDRulerColor},linerange=//!#1-//!End,includerangemarker=false,firstnumber=0]{lib/Snippets.ned}}
\newcommand{\inisnippet}[2]{%
    \lstinputlisting[language=inifile,rulecolor=\color{IniRulerColor},linerange=\#!#1-\#!End,includerangemarker=false,firstnumber=0]{lib/Snippets.ini}}

%%
%% some customization
%%
\setlength{\parindent}{0pt}
\setlength{\parskip}{1ex}

%%
%% Shortcuts
%%
\newcommand{\appendixchapter}{\chapter} %% html converter needs to know which chapters are appendices

\newcommand{\tbf}{\textbf} %% bold faced text
\newcommand{\ttt}{\texttt} %% type writer font text

\newcommand{\tab}{\hspace*{5mm}} %% tabulator settings

\newcommand{\new}{$^{New!}$}
\newcommand{\changed}{$^{Changed!}$}

%% Colordefinition for table header rows (requires package colortbl)
\newcommand{\tabheadcol}{\rowcolor[gray]{0.8}}

%%
%% Module parameters list
%%
\newenvironment{params}{\begin{itemize}}{\end{itemize}}
\newcommand{\param}[2]{\item \fpar{#1}: #2}

%%
%% Function/Class/Macro/Variable/Program/Parameter/Define names
%%
%% Write the names in type writer font and do an index entry
%% Allows word wrap by automatic hyphenation
%%
%% Usage: \ffunc{take()}
%%    or: \ffunc[take()]{take(obj)}
%% the second form uses the bracketed word for the index entry
%%

\newcommand{\protocol}[1]{%
    {#1}}

%% NED type names
\newcommand{\nedtype}[2][\DefaultOpt]{\def\DefaultOpt{#2}%
  \index{#1}%
  \texttt{\hyphenchar\font=`\-\relax#2}}

%% MSG type names
\newcommand{\msgtype}[2][\DefaultOpt]{\def\DefaultOpt{#2}%
  \index{#1}%
  \texttt{\hyphenchar\font=`\-\relax#2}}

%% Function names
\newcommand{\ffunc}[2][\DefaultOpt]{\def\DefaultOpt{#2}%
  \index{#1}%
  \texttt{\hyphenchar\font=`\-\relax#2}}

%% Class names
\newcommand{\cppclass}[2][\DefaultOpt]{\def\DefaultOpt{#2}%
  \index{#1}%
  \texttt{\hyphenchar\font=`\-\relax#2}}

%% Macro names
\newcommand{\fmac}[2][\DefaultOpt]{\def\DefaultOpt{#2}%
  \index{#1}%
  \texttt{\hyphenchar\font=`\-\relax#2}}

%% Variable names
\newcommand{\fvar}[2][\DefaultOpt]{\def\DefaultOpt{#2}%
  \index{#1}%
  \texttt{\hyphenchar\font=`\-\relax#2}}

%% Program names
\newcommand{\fprog}[2][\DefaultOpt]{\def\DefaultOpt{#2}%
  \index{#1}%
  \texttt{\hyphenchar\font=`\-\relax#2}}

%% Parameter names
\newcommand{\fpar}[2][\DefaultOpt]{\def\DefaultOpt{#2}%
  \index{#1}%
  \texttt{\hyphenchar\font=`\-\relax#2}}

%% Defines
\newcommand{\fdef}[2][\DefaultOpt]{\def\DefaultOpt{#2}%
  \index{#1}%
  \texttt{\hyphenchar\font=`\-\relax#2}}

%% NED/MSG properties
\newcommand{\fprop}[2][\DefaultOpt]{\def\DefaultOpt{#2}%
  \index{#1}%
  \texttt{\hyphenchar\font=`\-\relax#2}}

%% Keywords (NED, MSG)
\newcommand{\fkeyword}[2][\DefaultOpt]{\def\DefaultOpt{#2}%
  \index{#1}%
  \textbf{\texttt{\hyphenchar\font=`\-\relax#2}}}

%% Configuration options
\newcommand{\fconfig}[2][\DefaultOpt]{\def\DefaultOpt{#2}%
  \index{#1}%
  \textbf{\texttt{\hyphenchar\font=`\-\relax#2}}}

%% File names
\newcommand{\ffilename}[2][\DefaultOpt]{\def\DefaultOpt{#2}%
  \index{#1}%
  \texttt{\hyphenchar\font=`\-\relax#2}}

%% Signals
\newcommand{\fsignal}[2][\DefaultOpt]{\def\DefaultOpt{#2}%
  \index{#1}%
  \texttt{\hyphenchar\font=`\-\relax#2}}

\newcommand{\fgate}[1]{\texttt{\hyphenchar\font=`\-\relax#1}}

%% do not number subsubsections
%\setcounter{secnumdepth}{4}

% limit the depth of TOC
\setcounter{tocdepth}{2}

%%
%% Start of document
%%
\begin{document}

%% set the image type preference
\DeclareGraphicsExtensions{.pdf,.png}

\pagestyle{empty}
\pagenumbering{roman}
\include{title}
\cleardoublepage

%%\setcounter{page}{1}
%\newpage
%%\pagenumbering{roman}

%% \shorttableofcontents{Chapters}{0}
%% \cleardoublepage

\tableofcontents
\cleardoublepage

\pagestyle{fancy}
\pagenumbering{arabic}

\include{ch-introduction}
\cleardoublepage

\include{ch-usage}
\cleardoublepage

\include{ch-developing-models}
\cleardoublepage

\chapter{Network Interfaces}
\label{cha:network-interfaces}

\section{Overview}

%TODO: MAC address, op mode, duplex mode, data rate, transmission power, queue limits, FCS mode

In INET simulations, network interface modules are the primary means of
communication between network nodes. They represent the required
combination of software and hardware elements from an operating system
point-of-view. 

Network interfaces are implemented with OMNeT++ compound modules that
conform to the \nedtype{INetworkInterface} module interface. 
Network interfaces can be further categorized as wired and wireless;
they conform to the \nedtype{IWiredInterface} and \nedtype{IWirelessInterface}
NED types, respectively, which are subtypes of \nedtype{INetworkInterface}.

\section{Built-in Network Interfaces}

INET provides pre-assembled network interfaces for several standard
protocols, protocol tunneling, hardware emulation, etc. The following list
gives the most commonly used network interfaces.

\begin{itemize}
    \item \nedtype{EthernetInterface} represents an \protocol{Ethernet} interface
    \item \nedtype{PppInterface} is for wired links using \protocol{PPP}
    \item \nedtype{Ieee80211Interface} represents a Wifi (\protocol{IEEE 802.11}) interface
    \item \nedtype{Ieee802154Interface} represents a \protocol{IEEE 802.15.4} interface
    \item \nedtype{BMacInterface}, \nedtype{LMacInterface}, \nedtype{XMacInterface} provide 
      low-power wireless sensor MAC protocols along with a simple hypothetical PHY protocol
    \item \nedtype{TunInterface} is a tunneling interface that can be directly used by applications
    \item \nedtype{LoopbackInterface} provides local loopback within the network node
    \item \nedtype{ExtInterface} represents a real-world interface, suitable for hardware-in-the-loop simulations
\end{itemize}

\section{Anatomy of Network Interfaces}

Network interfaces in the INET Framework are OMNeT++ compound modules that
contain many more components than just the corresponding layer 2 protocol
implementation. Most of these components are optional, i.e. absent by default,
and can be added via configuration.

Typical ingredients are:

\begin{itemize}
    \item \emph{Layer 2 protocol implementation}. For some interfaces such as
      \nedtype{PppInterface} this is a single module; for others like Ethernet
      and Wifi it consists of separate modules for MAC, LLC, and possibly 
      other subcomponents. 
    \item \emph{PHY model}. Some interfaces also contain separate
      module(s) that implement the physical layer. For example, 
      \nedtype{Ieee80211Interface} contains a radio module.
    \item \emph{Output queue}. This module is optional and absent by default, 
      because most MAC protocol implementations already contain an internal queue
      which is more efficient to work with. The possibility to plug in an 
      external queue module allows one to experiment with different queueing policies
      and implement QoS, RED, etc.
    \item \emph{Traffic conditioners} allow traffic shaping and policing elements
      to be added to the interface, for example to implement a Diffserv router.
    \item \emph{Hooks} allow extra modules to be inserted in the incoming
      and outgoing paths of packets. 
\end{itemize}


\subsection{Internal vs External Output Queue}

Network interfaces usually have the external queue module defined with a
parametric type like this: 

\begin{ned}
queue: <queueType> like IOutputQueue if queueType != "";
\end{ned} 

When \fpar{queueType} is empty (this is the default), the external queue 
module is absent, and the MAC (or equivalent L2) protocol will use its 
internal queue object. Conceptually, the internal queue is of inifinite size, 
but for better diagnostics one can often specify a hard limit for the queue
length in a module parameter -- if this is exceeded, the simulation 
stops with an error.

When \fpar{queueType} is not empty, it must name a NED type that 
implements the \nedtype{IOutputQueue} interface. The external 
queue module model allows modeling a finite buffer, or implement
various queueing policies for QoS and/or RED.

The most frequently used module type for external queue is 
\nedtype{DropTailQueue}, a finite-size FIFO that drops overflowing 
packets). Other queue types that implement queueing policies can be 
created by assembling compound modules from DiffServ components 
(see chapter \ref{cha:diffserv}). An example of such compound
modules is \nedtype{DiffservQueue}.

An example ini file fragment that installs drop-tail queues of size 10
on PPP interfaces:

\begin{inifile}
**.ppp[*].queueType = "DropTailQueue"
**.ppp[*].queue.frameCapacity = 10
\end{inifile}

\subsection{Traffic Conditioners}

Many network interfaces contain optional traffic conditioner submodules
defined with parametric types, like this: 

\begin{ned}
ingressTC: <ingressTCType> like ITrafficConditioner if ingressTCType != "";
egressTC: <egressTCType> like ITrafficConditioner if egressTCType != "";
\end{ned}

Traffic conditioners allow one to implement the policing and shaping actions
of a Diffserv router. They are added to the input or output packets paths  
in the network interface. (On the output path they are added before the queue 
module.) 

Traffic conditioners must implement the \nedtype{ITrafficConditioner} module
interface. Traffic conditioners can be assembled from DiffServ components 
(see chapter \ref{cha:diffserv}). There is no preassembled traffic conditioner
in INET, but you can find some in the example simulations.

An example configuration with fictituous types:

\begin{inifile}
**.ppp[*].ingressTCType = "CustomIngressTC"
**.ppp[*].egressTCType = "CustomEgressTC"
\end{inifile}


\subsection{Hooks}

Several network interfaces allow extra modules to be inserted in the incoming
and outgoing paths of packets at the top of the netwok interface. 
Hooks are added as a submodule vector with parametric type, like this: 

\begin{ned}
outputHook[numOutputHooks]: <default("Nop")> like IHook if numOutputHooks>0;
inputHook[numInputHooks]: <default("Nop")> like IHook if numInputHooks>0;
\end{ned}

This allows any number of hook modules to be added. The hook modules 
are chained in their numeric order.

Modules inserted as hooks may act as probes (for measuring or recording
traffic) or as means of modifying or perturbing the packet flow for 
experimentation. Module types implementing the \nedtype{IHook} NED interface
include \nedtype{ThruputMeter}, \nedtype{Delayer}, \nedtype{OrdinalBasedDropper},
and \nedtype{OrdinalBasedDuplicator}. 

The following ini file fragment inserts two hook modules into the output
paths of PPP interfaces, a delayer and a throughput meter:

\begin{inifile}
**.ppp[*].numOutputHooks = 2
**.ppp[*].outputHook[0].typename = "Delayer"
**.ppp[*].outputHook[1].typename = "ThruputMeter"
**.ppp[*].outputHook[0].delay = 3ms
\end{inifile}



\section{The Interface Table}

Network nodes normally contain an \nedtype{InterfaceTable} module.
The interface table is a sort of registry of all the network interfaces
in the host. It does not send or receive messages, other modules access it
via C++ function calls. Contents of the interface table can also
be inspected e.g. in Qtenv.

Network interfaces register themselves in the interface table at the
beginning of the simulation. Registration is usually the task of the
MAC (or equivalent) module. 


\section{Wired Network Interfaces}

Wired interfaces have a pair of special purpose OMNeT++ gates which represent
the capability of having an external physical connection to another network
node (e.g. Ethernet port). In order to make wired communication work,
these gates must be connected with special connections which represent the
physical cable between the physical ports. The connections must use special
OMNeT++ channels (e.g. \nedtype{DatarateChannel}) which determine datarate
and delay parameters.

Wired network interfaces are compound modules that implement the 
\nedtype{IWiredInterface} interface. INET has the following
wired network interfaces. 

\subsection{PPP}

Network interfaces for point-to-point links (\nedtype{PppInterface}) are 
described in chapter \ref{cha:ppp}. They are typically used in routers.

\subsection{Ethernet}

Ethernet interfaces (\nedtype{EthernetInterface}), alongside with models 
of Ethernet devices such as switches and hubs, are described in chapter
\ref{cha:ethernet}.

\section{Wireless Network Interfaces}

Wireless interfaces use direct sending\footnote{OMNeT++ \ttt{sendDirect()} calls} 
for communication instead of links, so their compound modules do not have
output gates at the physical layer, only an input gate dedicated to receiving. 
Another difference from the wired case is that wireless interfaces 
require (and collaborate with) a \textit{transmission medium} module 
at the network level. The medium module represents the shared transmission 
medium (electromagnetic field or acoustic medium), is responsible for 
modeling physical effects like signal attenuation, and maintains 
connectivity information. Also, while wired interfaces can do without
explicit modeling of the physical layer, a PHY module is an indispensable
part of a wireless interface.

Wireless network interfaces are compound modules that implement the 
\nedtype{IWirelessInterface} interface. In the following sections we 
give an overview of the wireless interfaces available in INET.

\subsection{Generic Wireless Interface}

The \nedtype{WirelessInterface} compound module is a generic implementation
of \nedtype{IWirelessInterface}. In this network interface, the types of the
MAC protocol and the PHY layer (the radio) are parameters:

\begin{ned}
mac: <macType> like IMacProtocol;
radio: <radioType> like IRadio if radioType != "";
\end{ned}

There are specialized versions of \nedtype{WirelessInterface} where 
the MAC and the radio modules are fixed to a particular value. 
One example is \nedtype{BMacInterface}, which contains a \nedtype{BMac}
and an \nedtype{ApskRadio}.

\subsection{IEEE 802.11}

IEEE 802.11 or Wifi network interfaces (\nedtype{Ieee80211Interface}),
alongside with models of devices acting as access points (AP),
are covered in chapter \ref{cha:80211}.

\subsection{IEEE 802.15.4}

\nedtype{Ieee802154Interface} is covered in a separate chapter, see \ref{cha:802154}.

\subsection{Wireless Sensor Networks}

MAC protocols for wireless sensor networks (WSNs) and the corresponding
network interfaces are covered in chapter \ref{cha:sensor-macs}.

\subsection{CSMA/CA} 

\nedtype{CsmaCaMac} implements an imaginary CSMA/CA-based MAC protocol with
optional acknowledgements and a retry mechanism. With the appropriate settings,
it can approximate basic 802.11b ad-hoc mode operation.

\nedtype{CsmaCaMac} provides a lot of room for experimentation: 
acknowledgements can be turned on/off, and operation parameters like
inter-frame gap sizes, backoff behaviour (slot time, minimum and maximum 
number of slots), maximum retry count, header and ACK frame sizes, bit rate,
etc. can be configured via NED parameters.

\nedtype{CsmaCaInterface} interface is a \nedtype{WirelessInterface} with
the MAC type set to \nedtype{CsmaCaMac}. 

\subsection{Acking MAC}

Not every simulation requires a detailed simulation of the lower layers.
\nedtype{AckingWirelessInterface} is a highly abstracted wireless interface 
that offers simplicity for scenarios where Layer 1 and 2 effects can be 
completely ignored, for example testing the basic functionality of a 
wireless ad-hoc routing protocol.

\nedtype{AckingWirelessInterface} is a \nedtype{WirelessInterface} 
parameterized to contain a unit disk radio (\nedtype{UnitDiskRadio})
and a trivial MAC protocol (\nedtype{AckingMac}). 

The most important parameter \nedtype{UnitDiskRadio} accepts is the 
transmission range. When a radio transmits a frame, all other radios 
within transmission range are able to receive the frame correctly, 
and radios that are out of range will not be affected at all. 
Interference modeling (collisions) is optional, and it is recommended
to turn it off with \nedtype{AckingMac}.

\nedtype{AckingMac} implements a trivial MAC protocol that has packet
encapsulation and decapsulation, but no real medium access procedure. 
Frames are simply transmitted on the wireless channel as soon as the
transmitter becomes idle. There is no carrier sense, collision avoidance, 
or collison detection. \nedtype{AckingMac} also provides an optional 
out-of-band acknowledgement mechanism (using C++ function calls, 
not actual wirelessly sent frames), which is turned on by default.
There is no retransmission: if the acknowledgement does not arrive
after the first transmission, the MAC gives up and counts the packet 
as failed transmission. 

\subsection{Shortcut}

\nedtype{ShortcutMac} implements error-free ``teleportation'' of packets 
to the peer MAC entity, with some delay computed from a transmission 
duration and a propagation delay. The physical layer is completely bypassed.
The corresponding network interface type, \nedtype{ShortcutInterface},
does not even have a radio model.

\nedtype{ShortcutInterface} is useful for modeling wireless networks
where full connectivity is assumed, and Layer 1 and Layer 2 effects
can be completely ignored. 

\section{Special-Purpose Network Interfaces}

 
\subsection{Tunnelling}

\nedtype{TunInterface} is a virtual network interface that can be used 
for creating tunnels, but it is more powerful than that.
It lets an application-layer module capture packets sent to 
the TUN interface and do whatever it pleases with it (including
sending it to a peer entity in an UDP or plain IPv4 packet.)

To set up a tunnel, add an instance of \nedtype{TunnelApp} to 
the node, and specify the protocol (IPv4 or UDP) and the remote
endpoint of the tunnel (address and port) in parameters. 

TODO example: see examples/inet/tunnel

\subsection{Local Loopback}

\nedtype{LoopbackInterface} provides local loopback within the network node.

\subsection{External Interface}

\nedtype{ExtInterface} represents a real-world interface, suitable for 
hardware-in-the-loop simulations. External interfaces are explained in 
chapter \ref{cha:emulation}.

\section{Custom Network Interfaces}

It's also possible to build custom network interfaces, the following
example shows how to build a custom wireless interface.

\nedsnippet{WirelessInterfaceExample}{Wireless interface example}

The above network interface contains very simple hypothetical MAC and PHY
protocols. The MAC protocol only provides acknowledgment without other
services (e.g., carrier sense, collision avoidance, collision detection),
the PHY protocol uses one of the predefined APSK modulations for the whole
signal (preamble, header, and data) without other services (e.g.,
scrambling, interleaving, forward error correction).


%%% Local Variables:
%%% mode: latex
%%% TeX-master: "usman"
%%% End:



\cleardoublepage

\include{ch-packets}
\cleardoublepage

\chapter{Working with Sockets}
\label{cha:sockets}

\section{Overview}

The INET Socket API provides C++ abstractions over the standard OMNeT++ message
passing interface for several communication protocols. Sockets are most often
used by applications and routing protocols to acccess the corresponding protocol
services.

In the following sections, we introduce all INET sockets in detail, and we shed
light on many common usages through examples.

\section{General Rules}

Although sockets are protocol specific, INET provides an \cppclass{ISocket}
interface, which is implemtented by each one of them. This interface allows
general C++ code to handle all kinds of sockets.

\subsection*{Configuring Sockets}

Since sockets work with message passing under the hoods, they must be configured
to be able to send packets and commands on the right gate to the underlying
communication protocol.

\cppsnippet{SocketConfigureExample}{Socket configure example}

\subsection*{Identifying Sockets}

All sockets have an identifier which is unique within the network node. It is
assigned to them when the sockets are created. The identifier can be later
accessed with \ffunc{ISocket::getSocketId()}.

It is also passed along in \cppclass{SocketReq} and \cppclass{SocketInd} packet
tags. These tags allow identifying which socket \cppclass{Packet}s,
\cppclass{Request}s, and \cppclass{Indication}s belong to.

\subsection*{Closing Sockets}

Sockets must be closed before deleting them to release the underlying resources
allocated in the network.

\cppsnippet{UDPSocketConfigureExample}{UDP socket configure example}

\section{UDP Socket}

The \cppclass{UdpSocket} class provides an easy to use interface to send and
receive \protocol{UDP} datagrams. Applications can simply call the member
functions (\ffunc{bind}, \ffunc{connect}, \ffunc{send}, \ffunc{sendTo}, etc.) to
create and configure sockets, and to send and receive \protocol{UDP} datagrams.
They may use several \cppclass{UdpSocket} objects simulatenously.

The \cppclass{UdpSocket} automatically selects and stores the \fvar{socketId},
assembles and sends commands such as \fmac{UDP\_C\_BIND} to the \nedtype{Udp}
module, and can also help you deal with datagrams and indications arriving from
the \nedtype{Udp} module.

\subsection*{Configuring Sockets}

Since the \cppclass{UdpSocket} works with message passing under the hoods, it
must be configured to be able to send packets and commands on the right gate to
the underlying communication protocol.

\cppsnippet{UDPSocketConfigureExample}{UDP socket configure example}

For receiving \protocol{UDP} datagrams on a socket, it must also be bound to an
address and a port. Both the address and port is optional. If the address is
unspecified, than all \protocol{UDP} datagrams with any destination address are
received. If the port is -1, then an unused port is selected automatically by
the \nedtype{Udp} module. The address and port pair must be unique within the
same network node.

Here is a code fragment which binds to a specific local address and local port
to receive \protocol{UDP} datagrams:

\cppsnippet{UDPSocketBindExample}{UDP socket bind example}

There are several other socket options (e.g. broadcast, multicast groups, type
of service) which can also be configured using the \cppclass{UdpSocket} class.

\subsection*{Sending Datagrams}

After the socket has been configured, applications can send datagrams to a
remote address and port via a simple function call.

\cppsnippet{UDPSocketSendToExample}{UDP socket sendTo example}

If the application wants to send several datagrams, it can optionally connect to
the destination. The \protocol{UDP} protocol is in fact connectionless, so when
the \nedtype{Udp} module receives the connect request, it simply remembers the
remote address and port, and use it as default destination for later sends.
The application can call connect several times on the same socket.

\cppsnippet{UDPSocketSendExample}{UDP socket send example}

\subsection*{Receiving Datagrams}

Processing packets and indications which are received from the \nedtype{Udp}
module is pretty simple. The incoming message must be processed by the socket
where it belongs.

\cppsnippet{UDPSocketReceiveExample}{UDP socket receive example}

The \cppclass{UdpSocket} class deconstructs the message and uses the
\cppclass{UdpSocket::ICallback} interface to notify the application about
received data and error indications. The \cppclass{UdpSocket::ICallback}
interface contains only a few functions which are to be implemented by the
application.

\cppsnippet{UDPSocketCallbackInterfaceExample}{UDP socket callback interface
example}

\section{TCP Socket}

%The \cppclass{TcpSocket} C++ class is provided to simplify managing TCP connections
%from applications. \cppclass{TcpSocket} handles the job of assembling and sending
%command messages (OPEN, CLOSE, etc) to \nedtype{Tcp}, and it also simplifies
%the task of dealing with packets and notification messages coming from \nedtype{Tcp}.

\cppclass{TcpSocket} is a convenience class, to make it easier to manage TCP connections
from your application models. You'd have one (or more) \cppclass{TcpSocket} object(s)
in your application simple module class, and call its member functions
(bind(), listen(), connect(), etc.) to open, close or abort a TCP connection.

TCPSocket chooses and remembers the connId for you, assembles and sends command
packets (such as OPEN\_ACTIVE, OPEN\_PASSIVE, CLOSE, ABORT, etc.) to TCP,
and can also help you deal with packets and notification messages arriving
from TCP.

A session which opens a connection from local port 1000 to 10.0.0.2:2000,
sends 16K of data and closes the connection may be as simple as this
(the code can be placed in your \ffunc{handleMessage()} or
\ffunc{activity()}):

\subsection*{Configuring Sockets}

\cppsnippet{TCPSocketListenExample}{TCP socket listen example}

\subsection*{Sending Data}

\cppsnippet{TCPSocketSendExample}{TCP socket send example}

Dealing with packets and notification messages coming from TCP is somewhat
more cumbersome. Basically you have two choices: you either process those
messages yourself, or let TCPSocket do part of the job. For the latter,
you give TCPSocket a callback object on which it'll invoke the appropriate
member functions: \ffunc{socketEstablished()}, \ffunc{socketDataArrived()},
\ffunc{socketFailure()}, \ffunc{socketPeerClosed()},
etc (these are methods of \cppclass{TCPSocket::CallbackInterface}).,
The callback object can be your simple module class too.

This code skeleton example shows how to set up a TCPSocket to use the module
itself as callback object:

\subsection*{Receiving Data}

\cppsnippet{TCPSocketReceiveExample}{TCP socket receive example}

If you need to manage a large number of sockets (e.g. in a server
application which handles multiple incoming connections), the
\cppclass{TcpSocketMap} class may be useful. The following code
fragment to handle incoming connections is from the LDP module:

\cppsnippet{TCPSocketFindExample}{TCP socket find example}

\subsection*{Using Multiple Sockets}

\cppsnippet{TCPSocketMapExample}{TCP socket map example}

\section{SCTP Socket}

\section{IPv4 Socket}

\cppsnippet{IPv4SocketBindExample}{IPv4 socket bind example}

\cppsnippet{IPv4SocketSendExample}{IPv4 socket send example}

\cppsnippet{IPv4SocketReceiveExample}{IPv4 socket receive example}

\section{IPv6 Socket}

\ifdraft TODO
\section{Ethernet Socket}

\section{IEEE 802.11 Socket}
\fi

\section{Multiple Sockets}

If the application needs to manage a large number of sockets, for example in a
server application which handles multiple incoming TCP connections, the generic
\cppclass{SocketMap} class may be useful.

\cppsnippet{SocketFindExample}{Socket find example}


\cleardoublepage

\include{ch-cross-layer-communication}
\cleardoublepage

\include{ch-ppp}
\cleardoublepage

% last synchronized to 'dbc28949bf4332ac86d84b95705fbea9af4f84f7'
\chapter{The Ethernet Model}
\label{cha:ethernet}

TODO: 802.1d (STP, RSTP), 802.2 (LLC), ...

% TODO: comment numWirelessPorts in MacRelayUnitPP
% TODO: comment origByteLength in EtherFrame
% FIXME: wrong header length in EtherFrame.msg

\section{Overview}

Variations: 10Mb/s ethernet, fast ethernet, Gigabit Ethernet, Fast Gigabit Ethernet, full duplex

The Ethernet model contains a MAC model (\nedtype{EtherMac}), LLC model (\nedtype{EtherLlc}) as well
as a bus (\nedtype{EtherBus}, for modelling coaxial cable) and a hub (\nedtype{EtherHub}) model.
A switch model (\nedtype{EtherSwitch}) is also provided.

\begin{itemize}
  \item \nedtype{EtherHost} is a sample node with an Ethernet NIC;
  \item \nedtype{EtherSwitch}, \nedtype{EtherBus}, \nedtype{EtherHub} model switching hub, repeating hub and
        the old coxial cable;
  \item basic components of the model: \nedtype{EtherMac}, \nedtype{EtherLlc}/\nedtype{EtherEncap} module types,
        \nedtype{MacRelayUnit} (\nedtype{MACRelayUnitNP} and \nedtype{MACRelayUnitPP}), \nedtype{EtherFrame} message type,
        \cppclass{MacAddress} class
\end{itemize}


\section{Physical layer}

Stations on an Ethernet networks are connected by coaxial,
twisted pair or fibre cables. (Coaxial only has historical importance,
but is supported by INET anyway.) There are several cable types specified
in the standard.

In the INET framework, the cables are represented by connections.
The connections used in Ethernet LANs must be derived from
\nedtype{DatarateConnection} and should have their \fpar{delay} and
\fpar{datarate} parameters set.
The delay parameter can be used to model the distance between the
nodes. The datarate parameter can have four values:

\begin{itemize}
  \item \ttt{10Mbps} classic Ethernet
  \item \ttt{100Mbps} Fast Ethernet
  \item \ttt{1Gbps} Gigabit Ethernet
  \item \ttt{10Gbps} Fast Gigabit Ethernet
\end{itemize}


\subsection{EtherHub}

Ethernet hubs are a simple broadcast devices. Messages arriving on a port
are regenerated and broadcast to every other port.

The connections connected to the hub must have the same data rate.
Cable lengths should be reflected in the delays of the connections.

Messages are not interpreted by the \nedtype{EtherType} hub model in any way,
thus the hub model is not specific to Ethernet. Messages may
represent anything, from the beginning of a frame transmission to
end (or abortion) of transmission.

% TODO: model delay in hubs: class I device 140 bit time, class II device 92 bit time (for fast ethernet)

\subsection{EtherBus}

The \nedtype{EtherBus} component can model a common coaxial cable
found in early Ethernet LANs. The nodes are attached via taps at specific
positions on the cable. When a node sends a signal, it will propagate
along the cable in both directions at the given propagation speed.

The gates of the \nedtype{EtherBus} represent taps. The positions
of the taps are given by the \fpar{positions} parameter as a
space separated list of distances in metres. If there are more
gates then positions given, the last distance is repeated.
The bus component send the incoming message in one direction and
a copy of the message to the other direction (except at the ends).
The propagation delays are computed from the distances of the taps
and the \fpar{propagationSpeed} parameter.

Messages are not interpreted by the bus model in any way, thus the bus
model is not specific to Ethernet. Messages may represent anything, 
from the beginning of a frame transmission to end (or abortion) of transmission.

% FIXME #356 NED comment is wrong: data rate must not be zero!
% FIXME #354 default propagation speed is wrong (should be 2e8mps)
%            btw there is a hard coded propagation speed in EtherMACBase.cc


\section{Ethernet Interfaces}

\subsection{EthernetInterface}

The \nedtype{EthernetInterface} compound module implements the \nedtype{IWiredInterface}
interface. Complements \nedtype{EtherMac} and \nedtype{EtherEncap} with an output queue
for QoS and RED support. It also has configurable input/output filters as \nedtype{IHook}
components similarly to the \nedtype{PppInterface} module.

% TODO there is no IWiredNic with EtherLLC


\subsection{Ethernet MAC Layer}

The Ethernet MAC (Media Access Control) layer transmits the Ethernet frames on
the physical media. This is a sublayer within the data link layer. Because
encapsulation/decapsulation is not always needed (e.g. switches does not do
encapsulation/decapsulation), it is implemented in a separate modules
(\nedtype{EtherEncap} and \nedtype{EtherLlc}) that are part of the LLC layer.

\subsection{Implemented Standards}

The Ethernet model operates according to the following standards:

\begin{itemize}
  \item Ethernet: IEEE 802.3-1998
  \item Fast Ethernet: IEEE 802.3u-1995
  \item Full-Duplex Ethernet with Flow Control: IEEE 802.3x-1997
  \item Gigabit Ethernet: IEEE 802.3z-1998
\end{itemize}

Nowadays almost all Ethernet networks operate using full-duplex
point-to-point connections between hosts and switches. This means
that there are no collisions, and the behaviour of the MAC component
is much simpler than in classic Ethernet that used coaxial cables and
hubs. The INET framework contains two MAC modules for Ethernet:
the \nedtype{EtherMacFullDuplex} is simpler to understand and easier to extend,
because it supports only full-duplex connections. The \nedtype{EtherMac}
module implements the full MAC functionality including CSMA/CD, it
can operate both half-duplex and full-duplex mode.

\subsection*{Packets and frames}

The environment of the MAC modules is described by the \nedtype{IEtherMac}
module interface. Each MAC modules has gates to connect to the physical
layer (\ttt{phys\$i} and \ttt{phys\$o}) and to connect to the upper layer
(LLC module is hosts, relay units in switches): \ttt{upperLayerIn} and
\ttt{upperLayerOut}.

When a frame is received from the higher layers, it must be an
\msgtype{EtherFrame}, and with all protocol fields filled out
(including the destination MAC address). The source address, if left empty,
will be filled in with the configured \fpar{address} of the MAC.
% TODO document auto MAC address


Packets received from the network are \msgtype{EtherTraffic} objects.
They are messages representing inter-frame gaps (\msgtype{EtherPadding}),
jam signals (\msgtype{EtherJam}), control frames (\msgtype{EtherPauseFrame})
or data frames (all derived from \msgtype{EtherFrame}). Data frames
are passed up to the higher layers without modification.
In \fpar{promiscuous} mode, the MAC passes up all received frames;
otherwise, only the frames with matching MAC addresses and
the broadcast frames are passed up.

Also, the module properly responds to PAUSE frames, but never sends them
by itself -- however, it transmits PAUSE frames received from upper layers.
See section~\ref{subsec:pause_handling} for more info.

\subsection*{Queueing}

When the transmission line is busy, messages received from the upper layer
needs to be queued.

In routers, MAC relies on an external queue module (see \nedtype{OutputQueue}),
and requests packets from this external queue one-by-one. The name of the
external queue must be given as the \fpar{queueModule} parameer.
There are implementations of \nedtype{OutputQueue} to model finite buffer,
QoS and/or RED.

In hosts, no such queue is used, so MAC contains an internal
queue named \fvar{txQueue} to queue up packets waiting for transmission.
Conceptually, \fvar{txQueue} is of infinite size, but for better diagnostics
one can specify a hard limit in the \fpar{txQueueLimit} parameter -- if this is
exceeded, the simulation stops with an error.

\subsection*{PAUSE handling}
\label{subsec:pause_handling}

The 802.3x standard supports PAUSE frames as a means of flow
control. The frame contains a timer value, expressed as a multiple
of 512 bit-times, that specifies how long the transmitter should
remain quiet. If the receiver becomes uncongested before the
transmitter's pause timer expires, the receiver may elect to send
another PAUSE frame to the transmitter with a timer value of zero,
allowing the transmitter to resume immediately.

\nedtype{EtherMac} will properly respond to PAUSE frames it receives
(\msgtype{EtherPauseFrame} class),
however it will never send a PAUSE frame by itself. (For one thing,
it doesn't have an input buffer that can overflow.)

\nedtype{EtherMac}, however, transmits PAUSE frames received by higher layers,
and \nedtype{EtherLlc} can be instructed by a command to send a PAUSE frame to MAC.

% FIXME PAUSE frames should only be sent on full-duplex ethernet.
%       If a switch uses half-duplex mode to connect to hosts, it can ask sending hosts
%       to slow down their sending rates:
%       - force collisions with incoming frames
%       - make it appear as if the channel is busy
% FIXME PAUSE frames should have 0x8808 in the etherType field

\subsection*{Error handling}

If the MAC is not connected to the network ("cable unplugged"), it will
start up in "disabled" mode. A disabled MAC simply discards any messages
it receives. It is currently not supported to dynamically connect/disconnect
a MAC.

CRC checks are modeled by the \fvar{bitError} flag of the packets. Erronous
packets are dropped by the MAC.


%\subsection*{Auto-Negotiation}
% Ethernet Auto-Negotiation not supported



\subsection{EtherMacFullDuplex}

From the two MAC implementation \nedtype{EtherMacFullDuplex} is the simpler one,
it operates only in full-duplex mode (its \fpar{duplexEnabled} parameter fixed to
\ttt{true} in its NED definition). This module does not need to implement
CSMA/CD, so there is no collision detection, retransmission with exponential backoff,
carrier extension and frame bursting. Flow control works as described in section
\ref{subsec:pause_handling}.

% FIXME remove frameBursting from NED def, or fix it to false
%       currently setting it to 'true' has no effect

In the \nedtype{EtherMacFullDuplex} module,
packets arrived at the \ttt{phys\$i} gate are handled when their last bit received.

Outgoing packets are transmitted according to the following state diagram:

\begin{center}
\includegraphics{figures/EtherMACFullDuplex_txstates}
\end{center}

The \nedtype{EtherMacFullDuplex} module records two scalars in addition to the
ones mentioned earlier:
\begin{itemize}
\item \ttt{rx channel idle (\%)}: reception channel idle time
        as a percentage of the total simulation time
\item \ttt{rx channel utilization (\%)}: total reception
        time as a percentage of the total simulation time
\end{itemize}

\subsection{EtherMac}

Ethernet MAC layer implementing CSMA/CD. It supports both half-duplex and full-duplex operations;
in full-duplex mode it behaves as \nedtype{EtherMacFullDuplex}. In half-duplex  mode
it detects collisions, sends jam messages and retransmit frames upon collisions using
the exponential backoff algorithm. In Gigabit Ethernet networks it supports carrier
extension and frame bursting. Carrier extension can be turned off by setting the
\fpar{carrierExtension} parameter to \ttt{false}.

Unlike \nedtype{EtherMacFullDuplex}, this MAC module processes the incoming packets when their
first bit is received. The end of the reception is calculated by the MAC and
detected by scheduling a self message.

When frames collide the transmission is aborted -- in this case the transmitting
station transmits a jam signal. Jam signals are represented
by a \msgtype{EtherJam} message. The jam message contains the tree identifier
of the frame whose transmission is aborted. When the \nedtype{EtherMac} receives a jam
signal, it knows that the corresponding transmission ended in jamming and have
been aborted. Thus when it receives as many jams as collided frames, it can
be sure that the channel is free again. (Receiving a jam message marks the
beginning of the jam signal, so actually has to wait for the duration of the jamming.)

The operation of the MAC module can be schematized by the following state chart:

\begin{center}
\includegraphics{figures/EtherMAC_txstates}
\end{center}

The module generates these extra signals:
\begin{itemize}
\item \fsignal{collision} when collision starts (received a frame,
         while transmitting or receiving another one; or start to transmit while receiving a frame),
         the constant value 1
\item \fsignal{backoff} when jamming period ended and before waiting according to the
         exponential backoff algorith, the constant value 1
\end{itemize}

These scalar statistics are generated about the state of the line:
\begin{itemize}
  \item \ttt{rx channel idle (\%)} reception channel idle time (full duplex) or channel
         idle time (half-duplex), as a percentage of the total simulation time
  \item \ttt{rx channel utilization (\%)} total successful reception time (full-duplex) or total
         successful reception/transmission time (half duplex), as a percentage
         of the total simulation time
  \item \ttt{rx channel collision (\%)} total unsuccessful reception time, as a percentage
         of the total simulation time
  \item \ttt{collisions} total number collisions (same as count of \fsignal{collisionSignal})
  \item \ttt{backoffs} total number of backoffs (same as count of \fsignal{backoffSignal})
\end{itemize}

\subsection{EtherEncap}

The \nedtype{EtherEncap} module generates \msgtype{EthernetIIFrame} messages.

EtherFrameII

\subsection{EtherLlc}

TODO what it does


% document error conditions (causing error() calls in the code)

% FIXME handleRestransmission() comment is not true: // no beginSendFrames(), because end of jam signal sending will trigger it automatically
%       in case of inner queue, the queued msg is not transmitted
% FIXME should not enter PAUSE state when !duplexMode


\section{Switches}

Ethernet switches play an important role in modern Ethernet LANs. Unlike
passive hubs and repeaters, that work in the physical layer, the switches
operate in the data link layer and routes data frames between the connected
subnets.

While a hub repeats the data frames on each connected line, possibly causing
collisions, switches help to segment the network to small collision domains.
In modern Gigabit LANs each node is connected to the switch direclty
by full-duplex lines, so no collisions are possible. In this case the
CSMA/CD is not needed and the channel utilization can be high.

\subsection{MAC relay units}

INET framework ethernet switches are built from \nedtype{IMacRelayUnit}
components. Each relay unit has N input and output gates for sending/receiving
Ethernet frames. They should be connected to \nedtype{IEtherMac} modules.

Internally the relay unit holds a table for the destination address -> output
port mapping. When it receives a data frame it updates the table with the
source address->input port. The table can also be pre-loaded from a text file
while initializing the relay unit. The file name given as the \fpar{addressTableFile}
parameter. Each line of the file contains a hexadecimal MAC address and a decimal port
number separated by tabs. Comment lines beginning with '\#' are also allowed:

\begin{verbatim}
01 ff ff ff ff    0
00-ff-ff-ee-d1    1
0A:AA:BC:DE:FF    2
\end{verbatim}

% FIXME #352 addressTableSize is not checked in readAddressTable -> if overflown
%            then later check updateTableWithAddress has no effect
% FIXME format is wrong in the comment of readAddressTable()

The size of the lookup table is restricted by the \fpar{addressTableSize} parameter.
When the table is full, the oldest address is deleted. Entries are also deleted
if their age exceeds the duration given as the \fpar{agingTime} parameter.

If the destination address is not found in the table, the frame is broadcasted.
The frame is not sent to the same subnet it was received from, because the
target already received the original frame. The only exception if the frame
arrived through a radio channel, in this case the target can be out of range.
The port range 0..\fpar{numWirelessPorts}-1 are reserved for wireless connections.

The \nedtype{IMacRelayUnit} module is not a concrete implementation,
it just defines gates and parameters an \nedtype{IMacRelayUnit} should have.
Concrete inplementations add
capacity and performance aspects to the model (number of frames processed
per second, amount of memory available in the switch, etc.)
C++ implementations can subclass from the class \cppclass{MACRelayUnitBase}.

There are two versions of \nedtype{IMacRelayUnit}:

\begin{description}
  \item[\nedtype{MACRelayUnitNP}] models one or more CPUs with shared memory,
    working from a single shared queue.
  \item[\nedtype{MACRelayUnitPP}] models one CPU assigned to each incoming port,
    working with shared memory but separate queues.
\end{description}

In both models input messages are queued. CPUs poll messages from the queue
and process them in \fpar{processingTime}. If the memory usage exceeds
\fpar{bufferSize}, the frame will be dropped.

A simple scheme for sending PAUSE frames is built in (although
users will probably change it). When the buffer level goes
above a high watermark, PAUSE frames are sent on all ports.
The watermark and the pause time is configurable; use zero
values to disable the PAUSE feature.

% FIXME valid values for pauseTime: 0..0xFFFF
% FIXME ETHER_PAUSE_COMMAND_BYTES should be 4 in Ethernet.h (2bytes opcode + 2bytes pauseTime)
% FIXME PAUSE frame should not be sent on all ports probably
% TODO add lowWatermark, send PauseFrame(pauseUnits=0) to resume sending

The relay units collects the following statistics:

\begin{description}
\item[usedBufferBytes] memory usage as function of time
\item[processedBytes] count and length of processed frames
\item[droppedBytes] count and length of frames dropped caused by out of memory
\end{description}

% FIXME MACRelayUnitNP: no signals are generated, how does @statistic work in the ned file?

\subsection{EtherSwitch}

Model of an Ethernet switch containing a relay unit and multiple MAC units.

The duplexChannel attributes of the MACs must be set according to the
medium connected to the port; if collisions are possible (it's a bus or hub)
it must be set to false, otherwise it can be set to true.
By default it uses half duples MAC with CSMA/CD.

TODO STP, RSTP



%%% Local Variables:
%%% mode: latex
%%% TeX-master: "usman"
%%% End:

\cleardoublepage

\include{ch-environment}
\cleardoublepage

\include{ch-power}
\cleardoublepage

\chapter{The Physical Layer (Transceiver Modeling)}
\label{cha:physicallayer}

\section{Overview}

Wireless network interfaces contain a radio model component, which is
responsible for modeling the physical layer (PHY).\footnote{Wired network interfaces
could similarly contain an explicit PHY model. The reason they do not is that
wired links normally have very low error rates and simple observable behavior,
and there is usually not much to be gained from modeling the physical layer in detail.}
The radio model describes the physical device that is capable of transmitting
and receiving signals on the medium. 

Conceptually, a radio model relies on several sub-models:

\begin{itemize}
  \item antenna model
  \item transmitter model 
  \item receiver model 
  \item error model (as part of the receiver model)
  \item energy consumption model 
\end{itemize}

The antenna model is shared between the transmitter model and the receiver model.
The separation of the transmitter model and the receiver model allows 
asymmetric configurations. The energy consumer model is optional, and 
it is only used when the simulation of energy consumption is necessary.

TODO multiple implementations are provided for each model. For different
level of detail (abstract/fast versus detailed), different modeling strategy, etc.

TODO explain scalar, dimensional, and ``layered''

TODO different signal representations for models of different detail levels, etc.

\section{Generic Radio}

In INET, radio models implement the \nedtype{IRadio} module interface. 
A generic, often used implementation of \nedtype{IRadio} is the 
\nedtype{Radio} NED type. \nedtype{Radio} is an active compound module, 
that is, it has an associated C++ class that encapsulates the computations.

\nedtype{Radio} contains its antenna, transmitter, receiver and energy
consumer models as submodules with parametric types:

\begin{ned}
antenna: <antennaType> like IAntenna;
transmitter: <transmitterType> like ITransmitter;
receiver: <receiverType> like IReceiver;
energyConsumer: <energyConsumerType> like IEnergyConsumer 
    if energyConsumerType != "";
\end{ned}

The following sections describe the parts of the radio model.

\section{Components of a Radio}

\subsection{Antenna Models}

The antenna model describes the effects of the physical device which converts
electric signals into radio waves, and vice versa. This model captures the
antenna characteristics that heavily affect the quality of the communication
channel. For example, various antenna shapes, antenna size and geometry, antenna
arrays, and antenna orientation causes different directional or frequency
selectivity.

The antenna model provides a position and an orientation using a mobility model
that defaults to the mobility of the node. The main purpose of this model is to
compute the antenna gain based on the specific antenna characteristics and the
direction of the signal. The signal direction is computed by the medium from the
position and the orientation of the transmitter and the receiver. The following
list provides some examples:

\begin{itemize}
  \item \nedtype{IsotropicAntenna}: antenna gain is exactly 1 in any direction
  \item \nedtype{ConstantGainAntenna}: antenna gain is a constant determined by
    a parameter
  \item \nedtype{DipoleAntenna}: antenna gain depends on the direction according
    to the dipole antenna characteristics
  \item \nedtype{InterpolatingAntenna}: antenna gain is computed by linear
    interpolation according to a table indexed by the direction angles
\end{itemize}

\subsection{Transmitter Models}

The transmitter model describes the physical process which converts packets into
electric signals. In other words, this model converts an L2 frame into a signal
that is transmitted on the medium. The conversion process and the representation
of the signal depends on the level of detail and the physical characteristics
of the implemented protocol.

There are two main levels of detail (or modeling depths):
 
\begin{itemize}
\item In the \textit{flat model}, the transmitter model skips the symbol domain 
and the sample domain representations, and it directly creates the analog domain 
representation. The bit domain representation is reduced to the bit length of 
the packet, and the actual bits are ignored.

\item In the \textit{layered model}, the conversion process involves various 
processing steps such as packet serialization, forward error correction encoding, 
scrambling, interleaving, and modulation. This transmitter model requires 
significantly more computation, but it produces accurate bit domain, 
symbol domain, and sample domain representations.
\end{itemize}

Some of the transmitter types available in INET:

\begin{itemize}
  \item \nedtype{UnitDiskTransmitter}
  \item \nedtype{ApskScalarTransmitter}
  \item \nedtype{ApskDimensionalTransmitter}
  \item \nedtype{ApskLayeredTransmitter}
  \item \nedtype{Ieee80211ScalarTransmitter}
  \item \nedtype{Ieee80211DimensionalTransmitter}
\end{itemize}

TODO scalar radio parameters:
    power
    frquency
    bandwidth
    ...

TODO dimensional parameterization: 

 ApskDimensionalTransmitter:
        string dimensions = default("time");                // dimensions of power: time and/or frequency
        string timeGains = default("0% 0dB 100% 0dB");      // sequence of time and gain pairs; time is in [%] or [s], negative time measures from the end; gain is in [dB] or [0..1]; default value is a flat signal
        string frequencyGains = default("0% 0dB 100% 0dB"); // sequence of frequency and gain pairs; frequency is in [%] or [Hz], negative frequency measures from the end; gain is in [dB] or [0..1]; default value is a flat signal
        string interpolationMode @enum("linear", "sample-hold") = default("sample-hold");


\subsection{Receiver Models}

The receiver model describes the physical process which converts electric
signals into packets. In other words, this model converts a reception, along
with an interference computed by the medium model, into a MAC packet and a
reception indication.

For a packet to be received successfully, reception must be \textit{possible}
(based on reception power, bandwidth, modulation scheme and other characteristics),
it must be \textit{attempted} (i.e. the receiver must synchronize itself on
the preamble and start receiving), and it must be \textit{successful} 
(as determined by the error model and the simulated part of the signal decoding).

In the \textit{flat model}, the receiver model skips the sample domain, the symbol domain,
and the bit domain representations, and it directly creates the packet domain
representation by copying the packet from the transmission. It uses the error
model to decide whether the reception is successful.

In the \textit{layered model}, the conversion process involves various processing steps
such as demodulation, descrambling, deinterleaving, forward error correction
decoding, and deserialization. This reception model requires much more
computation than the flat model, but it produces accurate sample domain, 
symbol domain, and bit domain representations.

Some of the receiver types available in INET:

\begin{itemize}
  \item \nedtype{UnitDiskReceiver}
  \item \nedtype{ApskScalarReceiver}
  \item \nedtype{ApskDimensionalReceiver}
  \item \nedtype{ApskLayeredReceiver}
  \item \nedtype{Ieee80211ScalarReceiver}
  \item \nedtype{Ieee80211DimensionalReceiver}
\end{itemize}


\subsection{Error Models}

Determining reception errors is a crucial part of the reception process.
There are often several different statistical error models in the literature
even for a particular physical layer. In order to support this diversity, the
error model is a separate replaceable component of the receiver. 

The error model describes how the signal to noise ratio affects the amount of
errors at the receiver. The main purpose of this model is to determine whether
the received packet has errors or not. It also computes various physical
layer indications for higher layers such as packet error rate, bit error rate,
and symbol error rate. For the layered reception model it needs to compute the
erroneous bits, symbols, or samples depending on the lowest simulated physical
domain where the real decoding starts. The error model is optional (if omitted,
all receptions are considered successful.)

The following list provides some examples:

\begin{itemize}
  \item \nedtype{StochasticErrorModel}: simplistic error model with constant
    symbol/bit/packet error rates as parameters; suitable for testing. 
  \item \nedtype{ApskErrorModel} 
  \item \nedtype{Ieee80211NistErrorModel}, \nedtype{Ieee80211YansErrorModel}, 
    \nedtype{Ieee80211BerTable\-Error\-Model}: various error models for IEEE 802.11
    network interfaces.
\end{itemize}

\subsection{Power Consumption Models}

A substantial part of the energy consumption of communication devices comes from
transmitting and receiving signals. The energy consumer model describes how the
radio consumes energy depending on its activity. This model is optional (if
omitted, energy consumption is ignored.) 

The following list provides some examples:

\begin{itemize}
  \item \nedtype{StateBasedEpEnergyConsumer}: power consumption is
    determined by the radio state (a combination of radio mode, 
    transmitter state and receiver state), and specified in 
    parameters like \fpar{receiverIdlePowerConsumption} and 
    \fpar{receiverReceivingDataPowerConsumption}, in watts.
  \item \nedtype{StateBasedCcEnergyConsumer}: similar to the previous
    one, but consumption is given in amp\`eres.
\end{itemize}

\section{Layered Radio Models}

In layered radio models, the transmitter and receiver models are split
to several stages to allow more fine-grained modeling. 

For transmission, processing steps such as packet serialization, 
forward error correction (FEC) encoding, scrambling, interleaving, and 
modulation are explicitly modeled. Reception involves the inverse
operations: demodulation, descrambling, deinterleaving, 
FEC decoding, and deserialization.

In layered radio models, these processing steps are encapsulated
in four stages, represented as four submodules in both the 
transmitter and receiver model: 

\begin{enumerate}
  \item \textit{Encoding and Decoding} describe how the packet domain 
    signal representation is converted into the bit domain, and vice versa.
  \item \textit{Modulation and Demodulation} describe how the bit domain
    signal representation is converted into the symbol domain, and vice versa.
  \item \textit{Pulse Shaping and Pulse Filtering} describe how the 
    symbol domain signal representation is converted into the sample domain, 
    and vice versa.
  \item \textit{Digital Analog and Analog Digital Conversion} describe 
    how the sample domain signal representation is converted into the 
    analog domain, and vice versa.
\end{enumerate}

In layered radio transmitters and receivers such as \nedtype{ApskLayeredTransmitter}
and \nedtype{ApskLayeredReceiver}, these submodules have parametric
types to make them replaceable. This provides immense freedom for 
experimentation.

\section{Notable Radio Models}

The \nedtype{Radio} module has several specialized versions derived
from it, where certain submodule types and parameters are set to fixed values.
This section describes some of the frequently used ones.

The radio can be replaced in wireless network interfaces by setting the
\fpar{radioType} parameter, like in the following ini file fragment.
  
\begin{inifile}
**.wlan[*].radioType = "UnitDiskRadio"
\end{inifile}

However, be aware that not all MAC protocols can be used with all radio models,
and that some radio models require a matching transmission medium module. 

\subsection{UnitDiskRadio}

\nedtype{UnitDiskRadio} provides a very simple but fast and predictable 
physical layer model. It is the implementation (with some extensions)
of the \textit{Unit Disk Graph} model, which is widely used 
for the study of wireless ad-hoc networks.
\nedtype{UnitDiskRadio} is applicable if network nodes need 
to have a finite communication range, but physical effects 
of signal propagation are to be ignored.

\nedtype{UnitDiskRadio} allows three radii to be given as parameters,
instead of the usual one: communication range, interference range, and
detection range. One can also turn off interference modeling 
(meaning that signals colliding at a receiver will all be received 
correctly), which is sometimes a useful abstraction.

\nedtype{UnitDiskRadio} needs to be used together with a special physical
medium model, \nedtype{UnitDiskRadioMedium}.

The following ini file fragment shows an example configuration.

TODO wtf about those 0 meters???

\begin{inifile}
*.radioMediumType = "UnitDiskRadioMedium"
*.host[*].wlan[*].radioType = "UnitDiskRadio"
*.host[*].wlan[*].radio.transmitter.bitrate = 2Mbps
*.host[*].wlan[*].radio.transmitter.preambleDuration = 0s
*.host[*].wlan[*].radio.transmitter.headerLength = 100b
*.host[*].wlan[*].radio.transmitter.communicationRange = 100m
*.host[*].wlan[*].radio.transmitter.interferenceRange = 0m    
*.host[*].wlan[*].radio.transmitter.detectionRange = 0m
*.host[*].wlan[*].radio.receiver.ignoreInterference = true
\end{inifile}

As a side note, if modeling full connectivity and ignoring 
interference is required, then \nedtype{ShortcutInterface} 
provides an even simpler and faster alternative.

\subsection{APSK Radio}

APSK radio models provide a hypothetical radio that simulates 
one of the well-known ASP, PSK and QAM modulations. 
(APSK stands for Amplitude and Phase-Shift Keying.)

APSK radio has scalar/dimensional, and flat/layered variants.
The flat variants, \nedtype{ApskScalarRadio} and \nedtype{ApskDimensionalRadio}
model frame transmissons in the selected modulation scheme
but without utilizing other techniques such as forward error 
correction (FEC), interleaving, spreading, etc. These radios
require matching medium models, \nedtype{ApskScalarRadioMedium}
and \nedtype{ApskDimensionalRadioMedium}.

The layered versions, \nedtype{ApskLayeredScalarRadio} 
and \nedtype{ApskLayeredDimensionalRadio} can not only
model the processing steps missing from their simpler counterparts,
they also feature configurable level of detail: the transmitter
and receiver modules have \fpar{levelOfDetail} parameters that 
control which domains are actually simulated.
These radio models must be used in conjuction with
\nedtype{ApskLayeredScalarRadioMedium} and 
\nedtype{ApskLayeredDimensionalRadioMedium}, respectively.

TODO ApskLayeredScalarRadio and ApskLayeredDimensionalRadio types are missing, actually create them!!! 

TODO limitations for usage in real-world protocol models

TODO example: 1 for flat!

\begin{inifile}
TODO
\end{inifile}


TODO fragment for a layered one!

\begin{inifile}
## Iteration
**.wlan[*].radio.**.levelOfDetail = ${detail="packet", "bit", "symbol"}
**.wlan[*].radio.**.modulation = ${modulation="BPSK", "QPSK", "QAM-16", "QAM-64"}
**.wlan[*].radio.**.fecType = ${fecType="", "ConvolutionalCoder"}
**.bitrate = ${bitrate=$fecType == "" ? 36Mbps : 18Mbps} # we want to have the same 36Mbps gross bitrate (applying 1/2 code rate) 

## Transmitter
**.wlan[*].radio.transmitterType = "ApskLayeredTransmitter"
**.wlan[*].radio.transmitter.encoderType = "ApskEncoder"
**.wlan[*].radio.transmitter.modulatorType = "ApskModulator"

# scrambler
#**.wlan[*].radio.transmitter.scramblerType = "TODO"
**.wlan[*].radio.transmitter.scrambler.seed = "1011101"
**.wlan[*].radio.transmitter.scrambler.generatorPolynomial = "0001001"

# FEC
**.wlan[*].radio.transmitter.encoder.fecEncoder.transferFunctionMatrix = "1 3"
**.wlan[*].radio.transmitter.encoder.fecEncoder.constraintLengthVector = "2"
**.wlan[*].radio.transmitter.encoder.fecEncoder.puncturingMatrix = "1; 1"
**.wlan[*].radio.transmitter.encoder.fecEncoder.punctureK = 1
**.wlan[*].radio.transmitter.encoder.fecEncoder.punctureN = 2

# interleaver
# **.wlan[*].radio.transmitter.encoder.interleaverType = "TODO"

## Receiver
**.wlan[*].radio.receiverType = "ApskLayeredReceiver"
**.wlan[*].radio.receiver.errorModelType = "ApskLayeredErrorModel"
**.wlan[*].radio.receiver.decoderType = "ApskDecoder"
**.wlan[*].radio.receiver.demodulatorType = "ApskDemodulator"

# descrambler
#**.wlan[*].radio.receiver.scramblerType = "TODO"
**.wlan[*].radio.receiver.descrambler.seed = "1011101"
**.wlan[*].radio.receiver.descrambler.generatorPolynomial = "0001001"

# FEC
**.wlan[*].radio.receiver.decoder.fecDecoder.transferFunctionMatrix = "1 3"
**.wlan[*].radio.receiver.decoder.fecDecoder.constraintLengthVector = "2"
**.wlan[*].radio.receiver.decoder.fecDecoder.puncturingMatrix = "1; 1"
**.wlan[*].radio.receiver.decoder.fecDecoder.punctureK = 1
**.wlan[*].radio.receiver.decoder.fecDecoder.punctureN = 2

# Deinterleaver
# **.wlan[*].radio.receiver.decoder.deinterleaverType = "TODO"
\end{inifile}

\subsection{IEEE 802.11 Radios}

TODO why 802.11 needs specialized models 

\subsection{IEEE 802.15.4 Radios}

TODO why 802.15.4 needs specialized models 

\subsection{UWB-IR Radios}

TODO what is it


%%% Local Variables:
%%% mode: latex
%%% TeX-master: "usman"
%%% End:


\cleardoublepage

\include{ch-80211}
\cleardoublepage

\chapter{Node Mobility}
\label{cha:mobility}

\section{Overview}

In order to simulate ad-hoc wireless networks, it is important to model the
motion of mobile network nodes. Received signal strength, signal
interference, and channel occupancy depend on the distances between nodes.
The selected mobility models can significantly influence the results of the
simulation (e.g. via packet loss rates).

A mobility model describes position and orientation over time in a 3D
Euclidean coordinate system. Its main purpose is to provide position,
velocity and acceleration, and also angular position, angular velocity,
and angular acceleration data as three-dimensional quantities at the 
current simulation time.

In INET, a mobility model is most often an OMNeT++ simple module
implementing the motion as a C++ algorithm. Although most models have a few
common parameters (e.g. for initial positioning), they always come with
their own set of parameters. Some models support geographic positioning to
ease the configuration of map based scenarios.

Mobility models be \textit{single} or \textit{group} mobility models.
Single mobility models describe the motion of entities independent of each other. 
Group mobility models provide such a motion where group members are dependent
on each other.

Mobility models can also be categorized as \textit{trace-based}, 
\textit{deterministic}, \textit{stochastic}, and \textit{combining} models.

TODO:

initial positioning vs movement (positioning over time) -- example!

how to create initial layout + independent movements after

model controls what: position, orientation, or both (example: combining facing
with movement)

scope: single or group  (making group mobility using superpositioning and
"attached")

simple module or compound mobility, i.e. explain how use combining mobility

method of configuration (ini+ned, vs. xml)  "some models use XML or other files"



\subsection*{Using Mobility Models}

In order for a mobility model to actually have an effect on the motion of a network node,
the mobility model needs to be included as a submodule in the compound module of the
network node. By default, a transceiver antenna within a network node uses
the same mobility model as the node itself, but that is completely optional.
For example, it is possible to model a vehicle facing forward while moving 
on a road that contains multiple transceiver antennas at different relative 
locations with different orientations.

\subsection*{The Playground}

Many mobility models allow the user to define a cubic volume that the node 
can not leave. The volume is configured by setting the \fpar{constraintAreaX}, 
\fpar{constraintAreaY}, \fpar{constraintAreaZ},
\fpar{constraintAreaWidth}, \fpar{constraintAreaHeight} and
\fpar{constraintAreaDepth} parameters.

If the \fpar{initFromDisplayString} parameter, the initial position is taken from
the display string. Otherwise, the position can be given in the \fpar{initialX},
\fpar{initialY} and \fpar{initialZ} parameters. If neither of these parameters
are given, a random initial position is choosen within the contraint area.

When the node reaches the boundary of the constraint area, the mobility
component has to prevent the node to exit. Many mobility models offer the 
following policies:

\begin{itemize}
  \item reflect of the wall
  \item reappear at the opposite edge (torus area)
  \item placed at a randomly chosen position of the area
  \item stop the simulation with an error
\end{itemize}


\section{Built-In Mobility Models}

\subsection{List of Mobility Models}

The following, potentially list contains the mobility models available in INET.
Nearly all of these models als single mobility models; group mobility can be 
implemented e.g. with combining other mobility models.

\subsubsection*{Stationary}

Stationary models only define position (and orientation), but no motion.

\begin{itemize}
    \item \nedtype{StationaryMobility} provides deterministic and random positioning.
    \item \nedtype{StaticGridMobility} places several mobility models in a rectangular grid.
    \item \nedtype{StaticConcentricMobility} places several models in a set of concentric circles.
\end{itemize}

\subsubsection*{Deterministic}

Deterministic mobility models use non-random mathematical models for describing motion.

\begin{itemize}
    \item \nedtype{LinearMobility} moves linearly with a constant speed or constant acceleration.
    \item \nedtype{CircleMobility} moves around a circle parallel to the XY plane with constant speed.
    \item \nedtype{RectangleMobility} moves around a rectangular area parallel to the XY plane with constant speed.
    \item \nedtype{TractorMobility} moves similarly to a tractor on a field with a number of rows.
    \item \nedtype{VehicleMobility} moves similarly to a vehicle along a path especially turning around corners.
    \item \nedtype{TurtleMobility} moves according to an XML script written in a simple yet expressive LOGO-like programming language.
    \item \nedtype{FacingMobility} orients towards the position of another mobility model.
    \item \nedtype{RotatingMobility} rotates with a constant speed.
\end{itemize}

\subsubsection*{Trace-Based}

Trace-based mobility models replay recorded motion as observed in real life.

\begin{itemize}
    \item \nedtype{BonnMotionMobility} replays trace files of the BonnMotion scenario generator.
    \item \nedtype{Ns2MotionMobility} replays files of the CMU's scenario generator used in ns2.
    \item \nedtype{AnsimMobility} replays XML trace files of the ANSim (Ad-Hoc Network Simulation) tool.
\end{itemize}

\subsubsection*{Stochastic}

Stochastic or random mobility models use mathematical models involving random numbers.

\begin{itemize}
    \item \nedtype{RandomWaypointMobility} moves to random destination with random speed.
    \item \nedtype{GaussMarkovMobility} uses one parameter to vary the degree of randomness from linear to Brown motion.
    \item \nedtype{MassMobility} moves similarly to a mass with inertia and momentum.
    \item \nedtype{ChiangMobility} uses a probabilistic transition matrix to change the motion state.
\end{itemize}

\subsubsection*{Combining}

Combining mobility models are not mobility models per se, but instead, they
allow more complex motions to be formed from simpler ones via superposition 
and other ways. 

\begin{itemize}
        \item \nedtype{SuperpositioningMobility} model combines several other mobility models by summing them up. It allows creating group mobility by sharing a mobility model in each group member, separating initial positioning from positioning during the simulation, and separating positioning from orientation.
        \item \nedtype{AttachedMobility} models a mobility that is attached to another one at a given offset. Position, velocity and acceleration are all affected by the respective quantites and also the orientation of the referenced mobility.
\end{itemize}

\subsection{More Information on Some Mobility Models}

\begin{description}

\item[TractorMobility] Moves a tractor through a field with a certain
amount of rows. The following figure illustrates the movement of the
tractor when the \fpar{rowCount} parameter is 2. The trajectory follows
the segments in $1,2,3,4,5,6,7,8,1,2,3\ldots$ order. The area is configured
by the \fpar{x1}, \fpar{y1}, \fpar{x2}, \fpar{y2} parameters.

% TODO use constraint area instead of new x1,y1,x2,y2 parameters as in RectangleMobility

\begin{center}
\setlength{\unitlength}{0.5mm}
\begin{picture}(80,80)
\put(40,72){$1$} \put(10,70){\vector(1,0){30}} \put(10,70){\line(1,0){60}}
\put(72,55){$2$} \put(70,70){\vector(0,-1){15}} \put(70,70){\line(0,-1){30}}
\put(40,42){$3$} \put(70,40){\vector(-1,0){30}} \put(70,40){\line(-1,0){60}}
\put(5,25){$4$} \put(10,40){\vector(0,-1){15}} \put(10,40){\line(0,-1){30}}
\put(40,12){$5$} \put(10,10){\vector(1,0){30}} \put(10,10){\line(1,0){60}}
\put(72,25){$6$} \put(70,10){\vector(0,1){15}} \put(70,10){\line(0,1){30}}
\put(40, 33){$7$}
\put(5,55){$8$} \put(10,40){\vector(0,1){15}} \put(10,40){\line(0,1){30}}
\put(0,72){$(x_1,y_1)$} \put(65,2){$(x_2,y_2)$}
\end{picture}
\end{center}

\item[RandomWaypointMobility]

In the Random Waypoint mobility model the nodes move in line segments. For each
line segment, a random destination position (distributed uniformly over the
playground) and a random speed is chosen. You can define a speed as a variate
from which a new value will be drawn for each line segment; it is customary to
specify it as \ttt{uniform(minSpeed, maxSpeed)}. When the node reaches the
target position, it waits for the time \fpar{waitTime} which can also be defined as a
variate. After this time the the algorithm calculates a new random position, etc.

\item[GaussMarkovMobility] The Gauss-Markov model contains a tuning
parameter, that control the randomness in the movement of the node.
Let the magnitude and direction of speed of the node at the $n$th time step be
$s_n$ and $d_n$. The next speed and direction is computed as

$$ s_{n+1} = \alpha s_n + (1 - \alpha) \bar{s} +
             \sqrt{(1-\alpha^2)} s_{x_n} $$

$$ d_{n+1} = \alpha s_n + (1 - \alpha) \bar{d} +
             \sqrt{(1-\alpha^2)} d_{x_n} $$

where $\bar{s}$ and $\bar{d}$ are constants representing the mean value
of speed and direction as $n \to \infty$; and $s_{x_n}$ and $d_{x_n}$
are random variables with Gaussian distribution.

Totally random walk (Brownian motion) is obtained by setting $\alpha=0$,
while $\alpha=1$ results a linear motion.

To ensure that the node does not remain at the boundary of the constraint
area for a long time, the mean value of the direction ($\bar{d}$) modified
as the node enters the margin area. For example at the right edge of the
area it is set to 180 degrees, so the new direction is away from the edge.

% FIXME the GaussMarkovMobility module has only one variance parameter.
%       it should have separate speed and direction parameters

\item[MassMobility]

This is a random mobility model for a mobile host with
a mass. It is the one used in \cite{Perkins99optimizedsmooth}.

\begin{quote}
"An MH moves within the room according to the following pattern. It moves
along a straight line for a certain period of time before it makes a turn.
This moving period is a random number, normally distributed with average of
5 seconds and standard deviation of 0.1 second. When it makes a turn, the
new direction (angle) in which it will move is a normally distributed
random number with average equal to the previous direction and standard
deviation of 30 degrees. Its speed is also a normally distributed random
number, with a controlled average, ranging from 0.1 to 0.45 (unit/sec), and
standard deviation of 0.01 (unit/sec). A new such random number is picked
as its speed when it makes a turn. This pattern of mobility is intended to
model node movement during which the nodes have momentum, and thus do not
start, stop, or turn abruptly. When it hits a wall, it reflects off the
wall at the same angle; in our simulated world, there is little other
choice."
\end{quote}

This implementation can be parameterized a bit more, via the
\fpar{changeInterval}, \fpar{changeAngleBy} and \fpar{changeSpeedBy} parameters.
The parameters described above correspond to the following settings:

\begin{itemize}
\item changeInterval = normal(5, 0.1)
\item changeAngleBy = normal(0, 30)
\item speed = normal(avgSpeed, 0.01)
\end{itemize}

\item[ChiangMobility] Chiang's random walk movement model
(\cite{Chiang98wirelessnetwork}).

In this model, the state of the mobile node in each direction (x and y) can be:

\begin{itemize}
  \item 0: the node stays in its current position
  \item 1: the node moves forward
  \item 2: the node moves backward
\end{itemize}

The $(i,j)$ element of the state transition matrix determines the
probability that the state changes from $i$ to $j$:

$$ \left(
\begin{array}{ccc}
  0 & 0.5 & 0.5 \\
  0.3 & 0.7 & 0 \\
  0.3 & 0 & 0.7
\end{array}
\right) $$

\end{description}

\subsection{Replaying trace files}

\begin{description}

\item[BonnMotionMobility] Uses the native file format of
\href{http://bonnmotion.net}{BonnMotion}.

The file is a plain text file, where every line describes the motion
of one host. A line consists of one or more (t, x, y) triplets of real
numbers, like:

\begin{verbatim}
t1 x1 y1 t2 x2 y2 t3 x3 y3 t4 x4 y4 ...
\end{verbatim}

The meaning is that the given node gets to $(xk,yk)$ at $tk$. There's no
separate notation for wait, so x and y coordinates will be repeated there.

\item[Ns2MotionMobility] Nodes are moving according to the trace files used
in NS2. The trace file has this format:

\begin{verbatim}
# '#' starts a comment, ends at the end of line
$node_(<id>) set X_ <x> # sets x coordinate of the node identified by <id>
$node_(<id>) set Y_ <y> # sets y coordinate of the node identified by <id>
$node_(<id>) set Z_ <z> # sets z coordinate (ignored)
$ns at $time "$node_(<id>) setdest <x> <y> <speed>" # at $time start moving
towards <x>,<y> with <speed>
\end{verbatim}

The \nedtype{Ns2MotionMobility} module has the following parameters:

\begin{itemize}
  \item \fpar{traceFile} the Ns2 trace file
  \item \fpar{nodeId} node identifier in the trace file; -1 gets substituted by
  parent module's index
  \item \fpar{scrollX},\fpar{scrollY} user specified translation of the
  coordinates
\end{itemize}

% TODO cleaning the code (e.g. duplicated bounds check in setTargetPosition())
% TODO implement cached file access as in BonnMotionMobility

\item[ANSimMobility] reads trace files of the \href{http://www.ansim.info}{ANSim} Tool.

The nodes are moving along linear segments described by an XML trace file
conforming to this DTD:

\begin{XML}
<!ELEMENT mobility (position_change*)>
<!ELEMENT position_change (node_id, start_time, end_time, destination)>
<!ELEMENT node_id (#PCDATA)>
<!ELEMENT start_time (#PCDATA)>
<!ELEMENT end_time (#PCDATA)>
<!ELEMENT destination (xpos, ypos)>
<!ELEMENT xpos (#PCDATA)>
<!ELEMENT ypos (#PCDATA)>
\end{XML}

Parameters of the module:

\begin{itemize}
  \item \fpar{ansimTrace} the trace file
  \item \fpar{nodeId} the \verb!node_id! of this node, -1 gets substituted to
  parent module's index
\end{itemize}

\begin{note}
The \nedtype{AnsimMobility} module process only the \ttt{position\_{}change}
elements and it ignores the \ttt{start\_{}time} attribute. It starts the move
on the next segment immediately.
\end{note}


\end{description}


\subsection{TurtleMobility}

The \nedtype{TurtleMobility} module can be parametrized by a script file
containing LOGO-style movement commands in XML format. The content of 
the XML file should conform to the DTD in the \ffilename{TurtleMobility.dtd}
file in the source tree.

The file contains \ttt{movement} elements, each describing a trajectory.
The \ttt{id} attribute of the \ttt{movement} element can be used to
refer the movement from the ini file using the syntax:

\begin{inifile}
**.mobility.turtleScript = xmldoc("turtle.xml", "movements//movement[@id='1']")
\end{inifile}

The motion of the node is composed of uniform linear segments.
The \ttt{movement} elements may contain the the following commands as
elements (names in parens are recognized attribute names):

\begin{itemize}
\item \ttt{repeat(n)} repeats its content n times, or indefinitely if 
       the \ttt{n} attribute is omitted.
\item \ttt{set(x,y,speed,angle,borderPolicy)} modifies the state of the node.
      \ttt{borderPolicy} can be \ttt{reflect}, \ttt{wrap}, \ttt{placerandomly} 
      or \ttt{error}. 
\item \ttt{forward(d,t)} moves the node for $t$ time or to the $d$ distance
      with the current speed. If both $d$ and $t$ is given, then the current
      speed is ignored.
\item \ttt{turn(angle)} increase the angle of the node by $angle$ degrees.
\item \ttt{moveto(x,y,t)} moves to point $(x,y)$ in the given time. If
      $t$ is not specified, it is computed from the current speed.
\item \ttt{moveby(x,y,t)} moves by offset $(x,y)$ in the given time. If
      $t$ is not specified, it is computed from the current speed.
\item \ttt{wait(t)} waits for the specified amount of time.
\end{itemize}

Attribute values must be given without physical units, distances are assumed
to be given as meters, time intervals in seconds and speeds in meter per seconds.
Attibutes can contain expressions that are evaluated each time the
command is executed. The limits of the constraint area can be
referenced as \verb!$MINX!, \verb!$MAXX!, \verb!$MINY!, and \verb!$MAXY!.
Random number distibutions generate a new random number when evaluated,
so the script can describe random as well as deterministic scenarios.

To illustrate the usage of the module, we show how some mobility
models can be implemented as scripts.

RectangleMobility:

\begin{XML}
<movement>
    <set x="$MINX" y="$MINY" angle="0" speed="10"/>
    <repeat>
        <repeat n="2">
            <forward d="$MAXX-$MINX"/>
            <turn angle="90"/>
            <forward d="$MAXY-$MINY"/>
            <turn angle="90"/>
        </repeat>
    </repeat>
</movement>
\end{XML}

Random Waypoint:

\begin{XML}
<movement>
    <repeat>
        <set speed="uniform(20,60)"/>
        <moveto x="uniform($MINX,$MAXX)" y="uniform($MINY,$MAXY)"/>
        <wait t="uniform(5,10)">
    </repeat>
</movement>
\end{XML}

MassMobility:

\begin{XML}
<movement>
    <repeat>
        <set speed="uniform(10,20)"/>
        <turn angle="uniform(-30,30)"/>
        <forward t="uniform(0.1,1)"/>
    </repeat>
</movement>
\end{XML}


%%% Local Variables:
%%% mode: latex
%%% TeX-master: "usman"
%%% End:



\cleardoublepage

\include{ch-ipv4}
\cleardoublepage

\include{ch-ipv6}
\cleardoublepage

\include{ch-netfilter}
\cleardoublepage

\ifdraft TODO

\chapter{The UDP Model}
\label{cha:udp}

\section{The UDP module}

The state of the sockets are stored within the UDP module and the application
can configure the socket by sending command messages to the UDP module.
These command messages are distinguished by their kind and the type of their
control info. The control info identifies the socket and holds the parameters
of the command.

Applications don't have to send messages directly to the UDP module,
as they can use the \cppclass{UdpSocket} utility class, which encapsulates the messaging and
provides a socket like interface to applications.

\subsection{Sending UDP datagrams}

If the application want to send datagrams, it optionally can connect to the destination.
It does this be sending a message with UDP\_C\_CONNECT kind and \cppclass{UdpConnectCommand}
control info containing the remote address and port of the connection.
The UDP protocol is in fact connectionless, so it does not send any packets as a result
of the connect call. When the UDP module receives the connect request,
it simply remembers the destination address and port and use it as default destination
for later sends. The application can send several connect commands to the same socket.

% FIXME currently connect() or bind() is mandatory as the first command,
%       the application cannot send packets or set options otherwise

% FIXME connect() should allow unspecified dest address and -1 port (interpreted as disconnect())

For sending an UDP packet, the application should attach an \cppclass{UDPSendCommand}
control info to the packet, and send it to \nedtype{Udp}. The control info may contain
the destination address and port. If the destination address or port
is unspecified in the control info then the packet is sent to the connected target.

The \nedtype{Udp} module encapsulates the application's packet into an \msgtype{UDPPacket},
creates an appropriate IP control info and send it over ipOut or ipv6Out depending on
the destination address.

The destination address can be the IPv4 local broadcast address (255.255.255.255)
or a multicast address. Before sending broadcast messages, the socket must be configured
for broadcasting. This is done by sending an message to the UDP module. The message
kind is UDP\_C\_SETOPTION and its control info (an \cppclass{UdpSetBroadcastCommand})
tells if the broadcast is enabled. You can limit the multicast to the local network
by setting the TTL of the IP packets to 1. The TTL can be configured per socket,
by sending a message to the UDP with an \cppclass{UDPSetTimeToLive} control info
containing the value. If the node has multiple interfaces, the application can
choose which is used for multicast messages. This is also a socket option, the
id of the interface (as registered in the interface table) can be given in an
\cppclass{UdpSetMulticastInterfaceCommand} control info.

% FIXME currently sending broadcast messages is enabled without setting SO_BROADCAST to true,
%       this is not so in UNIX

% FIXME there should be a separate TTL for multicast (not used for unicast), default value is 1
%       see IP_MULTICAST_TTL in `man 7 ip`

\begin{note}
The \nedtype{Udp} module supports only local broadcasts (using the special 255.255.255.255 address).
Packages that are broadcasted to a remote subnet are handled as undeliverable messages.
\end{note}

If the UDP packet cannot be delivered because nobody listens on the destination port,
the application will receive a notification about the failure. The notification is
a message with UDP\_I\_ERROR kind having attached an \cppclass{UdpErrorIndication}
control info. The control info contains the local and destination address/port,
but not the original packet.

After the application finished using a socket, it should close it by sending a message
UDP\_C\_CLOSE kind and \cppclass{UdpCloseCommand} control info. The control info
contains only the socket identifier. This command frees the resources associated
with the given socket, for example its socket identifier or bound address/port.

\subsection{Receiving UDP datagrams}

Before receiving UDP datagrams applications should first ``bind'' to the given UDP port.
This can be done by sending a message with message kind UDP\_C\_BIND attached with an
\cppclass{UdpBindCommand} control info. The control info contains the socket identifier
and the local address and port the application want to receive UDP packets.
Both the address and port is optional. If the address is unspecified, than the UDP
packets with any destination address is passed to the application. If the port is
-1, then an unused port is selected automatically by the UDP module.
The localAddress/localPort combination must be unique.

When a packet arrives from the network, first its error bit is checked. Erronous messages
are dropped by the UDP component. Otherwise the application bound to the destination port
is looked up, and the decapsulated packet passed to it. If no application is bound to
the destination port, an ICMP error is sent to the source of the packet. If the socket is
connected, then only those packets are delivered to the application, that received from
the connected remote address and port.

The control info of the decapsulated packet is an \cppclass{UDPDataIndication}
and contains information about the source and destination address/port, the TTL,
and the identifier of the interface card on which the packet was received.

The applications are bound to the unspecified local address, then they receive any packets
targeted to their port. UDP also supports multicast and broadcast addresses; if they
are used as destination address, all nodes in the multicast group or subnet receives the packet.
The socket receives the broadcast packets only if it is configured for broadcast.
To receive multicast messages, the socket must join to the group of the multicast address.
This is done be sending the UDP module an UDP\_C\_SETOPTION message with
\cppclass{UdpJoinMulticastGroupsCommand} control info. The control info specifies the
multicast addresses and the interface identifiers. If the interface identifier is given
only those multicast packets are received that arrived at that interface.
The socket can stop receiving multicast messages if it leaves the multicast group.
For this purpose the application should send the UDP another UDP\_C\_SETOPTION
message in their control info (\cppclass{UdpLeaveMulticastGroupsCommand}) specifying
the multicast addresses of the groups.

% TODO clarify: multicast packets should not be delivered to connected sockets?

\subsection{Signals}

The \nedtype{Udp} module emits the following signals:
\begin{itemize}
  \item \fsignal{sentPk} when an UDP packet sent to the IP, the packet
  \item \fsignal{rcvdPk} when an UDP packet received from the IP, the packet
  \item \fsignal{passedUpPk} when a packet passed up to the application, the packet
  \item \fsignal{droppedPkWrongPort} when an undeliverable UDP packet received, the packet
  \item \fsignal{droppedPkBadChecksum} when an erronous UDP packet received, the packet
\end{itemize}

\fi



\cleardoublepage

\ifdraft TODO

\chapter{The TCP Models}
\label{cha:tcp}

\section{The TCP Module}

The \nedtype{Tcp} model relies on sending and receiving \cppclass{IPControlInfo} objects
attached to TCP segment objects as control info (see \ffunc{cMessage::setControlInfo()}).

The \nedtype{Tcp} module manages several \cppclass{TcpConnection} object each
holding the state of one connection. The connections are identified
by a connection identifier which is choosen by the application.
If the connection is established it can also be identified by
the local and remote addresses and ports. The TCP module simply
dispatches the incoming application commands and packets to
the corresponding object.

\subsection{TCP packets}
\label{subsec:tcp_packets}

The INET framework models the TCP header with the \msgtype{TcpHeader} message class.
This contains the fields of a TCP frame, except:
\begin{compactitem}
  \item \emph{Data Offset}: represented by \ffunc{cMessage::length()}
  \item \emph{Reserved}
  \item \emph{Checksum}: modelled by \ffunc{cMessage::hasBitError()}
  \item \emph{Options}: only EOL, NOP, MSS, WS, SACK\_PERMITTED, SACK and TS are possible
  \item \emph{Padding}
\end{compactitem}

The Data field can either be represented by (see \cppclass{TcpDataTransferMode}):
\begin{compactitem}
  \item encapsulated C++ packet objects,
  \item raw bytes as a \cppclass{ByteArray} instance,
  \item its byte count only,
\end{compactitem}
corresponding to transfer modes OBJECT, BYTESTREAM, BYTECOUNT resp.


\subsection{TCP commands}

The application and the TCP module communicates with each other
by sending \cppclass{cMessage} objects. These messages are specified
in the \ffilename{TCPCommand.msg} file.

The \cppclass{TCPCommandCode} enumeration defines the message kinds
that are sent by the application to the TCP:
\begin{itemize}
  \item TCP\_C\_OPEN\_ACTIVE: active open
  \item TCP\_C\_OPEN\_PASSIVE: passive open
  \item TCP\_C\_SEND: send data
  \item TCP\_C\_CLOSE: no more data to send
  \item TCP\_C\_ABORT: abort connection
  \item TCP\_C\_STATUS: request status info from TCP
\end{itemize}

Each command message should have an attached control info of type \cppclass{TcpCommand}.
Some commands (TCP\_C\_OPEN\_xxx, TCP\_C\_SEND) use subclasses.
The \cppclass{TcpCommand} object has a \fvar{connId} field that identifies the
connection locally within the application. \fvar{connId} is to be chosen by the
application in the open command.

When the application receives a message from the TCP, the message kind is
set to one of the \cppclass{TCPStatusInd} values:
\begin{itemize}
  \item TCP\_I\_ESTABLISHED: connection established
  \item TCP\_I\_CONNECTION\_REFUSED: connection refused
  \item TCP\_I\_CONNECTION\_RESET: connection reset
  \item TCP\_I\_TIME\_OUT: connection establish timer went off, or max retransmission count reached
  \item TCP\_I\_DATA: data packet
  \item TCP\_I\_URGENT\_DATA: urgent data packet
  \item TCP\_I\_PEER\_CLOSED: FIN received from remote TCP
  \item TCP\_I\_CLOSED: connection closed normally
  \item TCP\_I\_STATUS: status info
\end{itemize}

These messages also have an attached control info with \cppclass{TcpCommand}
or derived type (TCPConnectInfo, TCPStatusInfo, TCPErrorInfo).

% receive() calls are not modeled, incoming data passed to the application right away
% how accurate the modeling of the receiver window?

\subsection{TCP parameters}

The \nedtype{Tcp} module has the following parameters:
\begin{itemize}
  \item \fpar{advertisedWindow} in bytes, corresponds with the maximal receiver buffer capacity (Note: normally, NIC queues should be at least this size, default is  14*mss)
  \item \fpar{delayedAcksEnabled} delayed ACK algorithm (RFC 1122) enabled/disabled
  \item \fpar{nagleEnabled} Nagle's algorithm (RFC 896) enabled/disabled
  \item \fpar{limitedTransmitEnabled} Limited Transmit algorithm (RFC 3042) enabled/disabled (can be used for TCPReno/TCPTahoe/TCPNewReno/TCPNoCongestionControl)
  \item \fpar{increasedIWEnabled} Increased Initial Window (RFC 3390) enabled/disabled
  \item \fpar{sackSupport} Selective Acknowledgment (RFC 2018, 2883, 3517) support (header option) (SACK will be enabled for a connection if both endpoints support it)
  \item \fpar{windowScalingSupport} Window Scale (RFC 1323) support (header option) (WS will be enabled for a connection if both endpoints support it)
  \item \fpar{timestampSupport} Timestamps (RFC 1323) support (header option) (TS will be enabled for a connection if both endpoints support it)
  \item \fpar{mss} Maximum Segment Size (RFC 793) (header option, default is 536)
  \item \fpar{tcpAlgorithmClass} the name of TCP flavour

             Possible values are ``TCPReno'' (default), ``TCPNewReno'', ``TCPTahoe'', ``TCPNoCongestionControl'' and ``DumpTCP''.
             In the future, other classes can be written which implement Vegas, LinuxTCP  or other variants.
             See section \ref{sec:tcp_algorithms} for detailed description of implemented flavours.

             Note that TCPOpenCommand allows tcpAlgorithmClass to be chosen per-connection.

  \item \fpar{recordStats} if set to false it disables writing excessive amount of output vectors
\end{itemize}

\section{TCP connections}

Most part of the TCP specification is implemented in the
\cppclass{TcpConnection} class: takes care of the state machine,
stores the state variables (TCB), sends/receives SYN, FIN, RST, ACKs, etc.
TCPConnection itself implements the basic TCP ``machinery'',
the details of congestion control are factored out to
\cppclass{TcpAlgorithm} classes.

There are two additional objects the \cppclass{TcpConnection}
relies on internally: instances of \cppclass{TcpSendQueue} and
\cppclass{TcpReceiveQueue}. These polymorph classes manage the actual data stream,
so \cppclass{TcpConnection} itself only works with sequence number variables.
This makes it possible to easily accomodate need for various types of
simulated data transfer: real byte stream, "virtual" bytes (byte counts
only), and sequence of \cppclass{cMessage} objects (where every message object is
mapped to a TCP sequence number range).

\subsection{Data transfer modes}

Different applications have different needs how to represent
the messages they communicate with. Sometimes it is enough to
simulate the amount of data transmitted (``200 MB''), contents
does not matter. In other scenarios contents matters a lot.
The messages can be represented as a stream of bytes, but
sometimes it is easier for the applications to pass message
objects to each other (e.g. HTTP request represented by a
\msgtype{HTTPRequest} message class).

The TCP modules in the INET framework support 3 data transfer modes:

\begin{itemize}
  \item \ttt{TCP\_TRANSFER\_BYTECOUNT}: only byte counts are
        represented, no actual payload in \msgtype{TcpHeader}s.
        The TCP sends as many TCP segments as needed
  \item \ttt{TCP\_TRANSFER\_BYTESTREAM}: the application can pass
        byte arrays to the TCP. The sending TCP breaks down the bytes
        into MSS sized chunks and transmits them as the payload
        of the TCP segments. The receiving application can read the
        chunks of the data.
  \item \ttt{TCP\_TRANSFER\_OBJECT}: the application pass a
        \cppclass{cMessage} object to the TCP. The sending
        TCP sends as many TCP segments as needed according to
        the message length. The \cppclass{cMessage} object
        is also passed as the payload of the first segment. % check: first?
        The receiving application receives the object only
        when its last byte is received.
\end{itemize}

These values are defined in \ffilename{TCPCommand.msg} as
the \cppclass{TcpDataTransferMode} enumeration. The application
can set the data transfer mode per connection when the connection
is opened. The client and the server application must specify
the same data transfer mode.


\subsection{Opening connections}

Applications can open a local port for incoming connections by sending
the TCP a TCP\_C\_PASSIVE\_OPEN message. The attached control info
(an \cppclass{TcpOpenCommand}) contains the local address and port.
The application can specify that it wants to handle
only one connection at a time, or multiple simultanous connections. If the
\fvar{fork} field is true, it emulates the Unix accept(2) semantics: a new
connection structure is created for the connection (with a new \fvar{connId}),
and the connection with the old connection id remains listening.
If \fvar{fork} is false, then the first connection is accepted
(with the original \fvar{connId}),
and further incoming connections will be refused by the TCP by sending an RST segment.
The \fvar{dataTransferMode} field in \cppclass{TcpOpenCommand} specifies
whether the application data is transmitted as C++ objects, real bytes or byte
counts only. The congestion control algorithm can also be specified
on a per connection basis by setting \fvar{tcpAlgorithmClass} field to the
name of the algorithm.

The application opens a connection to a remote server by sending the TCP
a TCP\_C\_OPEN\_ACTIVE command. The TCP creates a \cppclass{TcpConnection}
object an sends a SYN segment. The initial sequence number selected according
to the simulation time: 0 at time 0, and increased by 1 in each 4$\mu$s.
If there is no response to the SYN segment, it retry after 3s, 9s, 21s and
45s. After 75s a connection establishment timeout (TCP\_I\_TIMEOUT) reported
to the application and the connection is closed.

When the connection gets established, TCP sends a TCP\_I\_ESTABLISHED
notification to the application. The attached control info
(a \cppclass{TcpConnectInfo} instance)
will contain the local and remote addresses and ports of the connection.
If the connection is refused by the remote peer (e.g. the port is not open),
then the application receives a TCP\_I\_CONNECTION\_REFUSED message.

\begin{note}
If you do active OPEN, then send data and close before the connection
has reached ESTABLISHED, the connection will go from SYN\_SENT to CLOSED
without actually sending the buffered data. This is consistent with
RFC 793 but may not be what you would expect.
\end{note}

\begin{note}
Handling segments with SYN+FIN bits set (esp. with data too) is
inconsistent across TCPs, so check this one if it is of importance.
\end{note}

\subsection{Sending Data}

The application can write data into the connection
by sending a message with TCP\_C\_SEND kind to the TCP.
The attached control info must be of type \cppclass{TCPSendCommand}.

The TCP will add the message to the \emph{send queue}.
There are three type of send queues corresponding to the
three data transfer mode. If the payload is transmitted as a message
object, then \cppclass{TCPMsgBasedSendQueue};
if the payload is a byte array then \cppclass{TCPDataStreamSendQueue};
if only the message lengths are represented then \cppclass{TCPVirtualDataSendQueue}
are the classes of send queues. The appropriate queue is created based
on the value of the \fpar{dataTransferMode} parameter of the Open command, no
further configuration is needed.

The message is handed over to the IP when there is
enough room in the windows. If Nagle's algorithm is
enabled, the TCP will collect 1 SMSS data and sends
them toghether.

\begin{note}
There is no way to set the PUSH and URGENT flags, when sending data.
\end{note}

% FIXME urgBit is never set
% FIXME model TCP_NODELAY, there is no PUSH flag in socket.send() (TCP_PUSH option ?)

\subsection{Receiving Data}

The TCP connection stores the incoming segments in the
\emph{receive queue}. The receive queue also has three flavours:
\cppclass{TCPMsgBasedRcvQueue}, \cppclass{TCPDataStreamRcvQueue}
and \cppclass{TCPVirtualDataRcvQueue}. The queue is created
when the connection is opened according to the \fvar{dataTransferMode}
of the connection.

Finite receive buffer size is modeled by the \fpar{advertisedWindow}
parameter. If receive buffer is exhausted (by out-of-order
segments) and the payload length of a new received segment
is higher than the free receiver buffer, the new segment will be dropped.
Such drops are recorded in \emph{tcpRcvQueueDrops} vector.

If the \emph{Sequence Number} of the received segment is the next
expected one, then the data is passed
to the application immediately. The \ffunc{recv()} call of
Unix is not modeled.

The data of the segment, which can be either a \cppclass{cMessage}
object, a \cppclass{ByteArray} object, or a simply byte count,
is passed to the application in a message that has
TCP\_I\_DATA kind.

% when the cMessage object is passed to the app? when last byte received?

\begin{note}
The TCP module does not handle the segments with PUSH or URGENT
flags specially. The data of the segment passed to the application
as soon as possible, but the application can not find out if that
data is urgent or pushed.
\end{note}

\subsection{RESET handling}

When an error occures at the TCP level, an RST segment is sent to
the communication partner and the connection is aborted.
Such error can be:
\begin{compactitem}
  \item arrival of a segment in CLOSED state
  \item an incoming segment acknowledges something not yet sent.
\end{compactitem}

The receiver of the RST it will abort the connection.
If the connection is not yet established, then the passive
end will go back to the LISTEN state and waits for another
incoming connection instead of aborting.

\subsection{Closing connections}

When the application does not have more data to send, it closes the
connection by sending a TCP\_C\_CLOSE command to the TCP. The TCP
will transmit all data from its buffer and in the last segment sets
the FIN flag. If the FIN is not acknowledged in time it will be
retransmitted with exponential backoff.

The TCP receiving a FIN segment will notify the application that
there is no more data from the communication partner. It sends
a TCP\_I\_PEER\_CLOSED message to the application containing
the connection identifier in the control info.

When both parties have closed the connection, the applications
receive a TCP\_I\_CLOSED message and the connection object is
deleted. (Actually one of the TCPs waits for $2 MSL$ before
deleting the connection, so it is not possible to reconnect
with the same addresses and port numbers immediately.)

\subsection{Aborting connections}

The application can also abort the connection. This means that
it does not wait for incoming data, but drops the data associated
with the connection immediately. For this purpose the application
sends a TCP\_C\_ABORT message specifying the connection identifier
in the attached control info. The TCP will send a RST to the
communication partner and deletes the connection object. The application
should not reconnect with the same local and remote addresses and
ports within MSL (maximum segment lifetime), because segments
from the old connection might be accepted in the new one.

\subsection{Status Requests}

Applications can get detailed status information about an existing
connection. For this purpose they send the TCP module a TCP\_C\_STATUS
message attaching an \cppclass{TcpCommand} info with the identifier
of the connection. The TCP will respond with a TCP\_I\_STATUS message
with a \cppclass{TcpStatusInfo} attachement. This control info
contains the current state, local and remote addresses and ports,
the initial sequence numbers, windows of the receiver and sender, etc.

% \section{TCP queues}
%
% Three queues belong to each TCP connection. The \emph{send queue} holds
% the segments not yet transmitted or not yet acknowledged.
% The \emph{receive queue} holds the segments received by the TCP,
% but not yet passed to the application. (This happens only when the segment
% is received out-of-order.). The \emph{retransmit queue} holds additional
% information about the segments in the send queue.
%
% As mentioned in section \ref{subsec:tcp_packets}, there are three methods
% to represent the application data in the TCP segment. Consequently the above
% queues comes in three flavours. If the payload is transmitted as a message
% object, then \cppclass{TCPMsgBasedRcvQueue} and \cppclass{TCPMsgBasedSendQueue};
% if the payload is a byte array then \cppclass{TCPDataStreamRcvQueue} and
% \cppclass{TCPDataStreamSendQueue}; if only the message lengths are represented
% then \cppclass{TCPVirtualDataRcvQueue} and \cppclass{TCPVirtualDataSendQueue}
% are the classes of receive/send queues. The appropriate queue is created based
% on the value of the \fpar{dataTransferMode} parameter of the Open command, no
% further configuration is needed. The retransmit queue is always an
% instance of \cppclass{TcpSackRexmitQueue}.
%
% The interfaces of the receive/send queues are defined by the
% \cppclass{TcpReceiveQueue} and \cppclass{TcpSendQueue} classes.
%
% % mapping segments into the sequence space
%

\section{TCP algorithms}
\label{sec:tcp_algorithms}

The \cppclass{TcpAlgorithm} object controls
retransmissions, congestion control and ACK sending: delayed acks, slow start,
fast retransmit, etc. They are all extends the \cppclass{TcpAlgorithm} class.
This simplifies the design of \cppclass{TcpConnection} and makes it a lot easier to
implement TCP variations such as Tahoe, NewReno, Vegas or LinuxTCP.

Currently implemented algorithm classes are \cppclass{TcpReno},
\cppclass{TcpTahoe}, \cppclass{TcpNewReno}, \cppclass{TcpNoCongestionControl}
and \cppclass{DumbTcp}. It is also possible to add new TCP variations
by implementing \cppclass{TcpAlgorithm}.

\includegraphics{figures/tcp_algorithms}

The concrete TCP algorithm class to use can be chosen per connection (in OPEN)
or in a module parameter.

\subsection{DumbTcp}

A very-very basic \cppclass{TcpAlgorithm} implementation, with hardcoded
retransmission timeout (2 seconds) and no other sophistication. It can be
used to demonstrate what happened if there was no adaptive
timeout calculation, delayed acks, silly window avoidance,
congestion control, etc. Because this algorithm does not
send duplicate ACKs when receives out-of-order segments,
it does not work well together with other algorithms.

\subsection{TcpBaseAlg}

The \cppclass{TcpBaseAlg} is the base class of the INET implementation
of Tahoe, Reno and New Reno. It implements basic TCP
algorithms for adaptive retransmissions, persistence timers,
delayed ACKs, Nagle's algorithm, Increased Initial Window
-- EXCLUDING congestion control. Congestion control
is implemented in subclasses.

\subsubsection*{Delayed ACK}

When the \fpar{delayedAcksEnabled} parameter is set to \fkeyword{true},
\cppclass{TcpBaseAlg} applies a 200ms delay before sending ACKs.

\subsubsection*{Nagle's algorithm}

When the \fpar{nagleEnabled} parameter is \fkeyword{true}, then
the algorithm does not send small segments if there is outstanding
data. See also \ref{subsec:trans_policies}.

\subsubsection*{Persistence Timer}

The algorithm implements \emph{Persistence Timer} (see \ref{subsec:flow_control}).
When a zero-sized window is received it starts the timer with 5s timeout.
If the timer expires before the window is increased, a 1-byte probe is
sent. Further probes are sent after 5, 6, 12, 24, 48, 60, 60, 60, ...
seconds until the window becomes positive.

\subsubsection*{Initial Congestion Window}

Congestion window is set to 1 SMSS when the connection is established.
If the \fpar{increasedIWEnabled} parameter is true, then the initial
window is increased to 4380 bytes, but at least 2 SMSS and at most 4 SMSS.
The congestion window is not updated afterwards; subclasses can
add congestion control by redefining virtual methods of the
\cppclass{TcpBaseAlg} class.

\subsubsection*{Duplicate ACKs}

The algorithm sends a duplicate ACK when an out-of-order
segment is received or when the incoming segment fills in all
or part of a gap in the sequence space.

\subsubsection*{RTO calculation}

Retransmission timeout ($RTO$) is calculated according to
Jacobson algorithm (with $\alpha=7/8$), and Karn's modification is also applied.
The initial value of the $RTO$ is 3s, its minimum is 1s,
maximum is 240s (2 MSL).

% FIXME according to RFC1222, MIN_REXMIT_TIMEOUT should be a fraction of second
%       to accomodate high speed LANs. In the linux kernel (net/tcp.h)
%       TCP_RTO_MIN is HZ/5 = 200ms. Consider 0ms lower bound.

\subsection{TCPNoCongestion}

TCP with no congestion control (i.e. congestion window kept very large).
Can be used to demonstrate effect of lack of congestion control.

% FIXME 65536 is not 'very large' nowadays, with window scaling
%       the receive window can be as large as 2^30 bytes.
%       Consequently the initial ssthresh is too small for Tahoe/Reno/NewReno,
%       Slow Start is stopped too early first time.

\subsection{TcpTahoe}

The \cppclass{TcpTahoe} algorithm class extends \cppclass{TcpBaseAlg}
with \emph{Slow Start}, \emph{Congestion Avoidance} and
\emph{Fast Retransmit} congestion control algorithms.
This algorithm initiates a \emph{Slow Start} when a packet
loss is detected.

\subsubsection*{Slow Start}

The congestion window is initially set to 1 SMSS or in case of
\fpar{increasedIWEnabled} is \fkeyword{true} to 4380 bytes
(but no less than 2 SMSS and no more than 4 SMSS). The window
is increased on each incoming ACK by 1 SMSS, so it is approximately
doubled in each RTT.

\subsubsection*{Congestion Avoidance}

When the congestion window exceeded $ssthresh$, the window
is increased by $SMSS^2/cwnd$ on each incoming ACK event, so
it is approximately increased by 1 SMSS per RTT.

\subsubsection*{Fast Retransmit}

When the 3rd duplicate ACK received, a packet loss is detected
and the packet is retransmitted immediately. Simultanously
the $ssthresh$ variable is set to half of the $cwnd$ (but at least 2 SMSS)
and $cwnd$ is set to 1 SMSS, so it enters slow start again.

Retransmission timeouts are handled the same way:
$ssthresh$ will be $cwnd/2$, $cwnd$ will be 1 SMSS.

\subsection{TcpReno}

The \cppclass{TcpReno} algorithm extends the behaviour \cppclass{TcpTahoe}
with \emph{Fast Recovery}. This algorithm can also use the information
transmitted in SACK options, which enables a much more accurate
congestion control.

\subsubsection*{Fast Recovery}

When a packet loss is detected by receiveing 3 duplicate ACKs,
$ssthresh$ set to half of the current window as in Tahoe. However
$cwnd$ is set to $ssthresh + 3*SMSS$ so it remains in congestion
avoidance mode. Then it will send one new segment for each incoming
duplicate ACK trying to keep the pipe full of data. This requires
the congestion window to be inflated on each incoming duplicate
ACK; it will be deflated to $ssthresh$ when new data gets
acknowledged.

However a hard packet loss (i.e. RTO events) cause a
slow start by setting $cwnd$ to 1 SMSS.

\subsubsection*{SACK congestion control}

This algorithm can be used with the SACK extension.
Set the \fpar{sackSupport} parameter to \fkeyword{true} to
enable sending and receiving \emph{SACK} options.

\subsection{TcpNewReno}

This class implements the TCP variant known as New Reno.
New Reno recovers more efficiently from multiple packet losses within one RTT
than Reno does.

It does not exit fast-recovery phase until all data which was out-standing
at the time it entered fast-recovery is acknowledged. Thus avoids
reducing the $cwnd$ multiple times.

\fi



\cleardoublepage

\include{ch-sctp}
\cleardoublepage

\chapter{Internet Routing}
\label{cha:routing}

\section{Overview}

INET Framework has models for several internet routing protocols, including
RIP, OSPF and BGP.

The easiest way to add routing to a network is to use the \nedtype{Router}
NED type for routers. \nedtype{Router} contains a conditional instance
for each of the above protocols. These submodules can be enabled by
setting the \ttt{hasRIP}, \ttt{hasOSPF} and/or \ttt{hasBGP} parameters to
\ttt{true}.

Example:

\begin{inifile}
**.hasRIP = true
\end{inifile}

There are also NED types called \nedtype{RipRouter}, \nedtype{OspfRouter},
\nedtype{BgpRouter}, which are all \nedtype{Router}s with appropriate
routing protocol enabled.

\section{RIP}
\label{sec:rip}

RIP (Routing Information Protocol) is a distance-vector routing protocol
which employs the hop count as a routing metric. RIP prevents routing loops
by implementing a limit on the number of hops allowed in a path from source
to destination.

The \nedtype{Rip} module implements distance vector routing as
specified in RFC 2453 (RIPv2) and RFC 2080 (RIPng). Configuration
can be specified in an XML file that can be specified in the
\ttt{ripConfig} parameter.

The configuration file specifies the per interface parameters.
Each \ttt{<interface>} element configures one or more interfaces;
the \ttt{hosts}, \ttt{names}, \ttt{towards}, \ttt{among} attributes
select the configured interfaces (in a similar way as with
\nedtype{Ipv4NetworkConfigurator} \ref{cha:network-autoconfiguration}).

Additional attributes:
\begin{itemize}
  \item \ttt{metric}: metric assigned to the link, default value is 1.
        This value is added to the metric of a learned route,
        received on this interface. It must be an integer in
        the [1,15] interval.
  \item \ttt{mode}: mode of the interface.
\end{itemize}

The mode attribute can be one of the following:
\begin{itemize}
  \item \ttt{'NoRIP'}: no RIP messages are sent or received on this interface.
  \item \ttt{'NoSplitHorizon'}: no split horizon filtering; send all routes to
        neighbors.
  \item \ttt{'SplitHorizon'}: do not sent routes whose next hop is the neighbor.
  \item \ttt{'SplitHorizonPoisenedReverse'} (default): if the next hop is the neighbor, then
  set the metric of the route to infinity.
\end{itemize}

The following example sets the link metric between router
\ttt{R1} and \ttt{RB} to 2, while all other links will have a metric of 1.
\begin{verbatim}
<RIPConfig>
  <interface among="R1 RB" metric="2"/>
  <interface among="R? R?" metric="1"/>
</RIPConfig>
\end{verbatim}

The \nedtype{Rip} module has the following parameters:
\begin{itemize}
  \item \ttt{mode}: either "RIPv2" (RFC 2453) or "RIPng" (RFC 2080)
  \item \ttt{routingTableModule}: path to the routing table module
        e.g. \ttt{'\^{}.ipv4.routingTable'}
  \item \ttt{ripConfig}: an XML configuration file containing per-interface parameters
\end{itemize}

The following example configures a \nedtype{Router} module to use RIPv2:
\begin{verbatim}
    **.hasRIP = true
    **.mode = "RIPv2"
    **.ripConfig = xmldoc("RIPConfig.xml")
\end{verbatim}

\section{OSPF}
\label{sec:ospf}

OSPF (Open Shortest Path First) is a routing protocol for IP networks.
It uses a link state routing (LSR) algorithm and falls into the group
of interior gateway protocols (IGPs), operating within a single
autonomous system (AS).

The \nedtype{Ospf} module implements the OSPF Version 2. Areas and routers
can be configured using an XML file.

\nedtype{OspfRouter} is a \nedtype{Router} with the OSPF protocol enabled.


\section{BGP}
\label{sec:bgp}

BGP (Border Gateway Protocol) is a standardized exterior gateway protocol
designed to exchange routing and reachability information among
autonomous systems (AS) on the Internet.

The \nedtype{Bgp} module implements BGP Version 4. The model implements
RFC 4271, with some limitations. Autonomous Systems and rules can be
configured in an XML file that can be specified in the \ttt{bgpConfig}
parameter.

\begin{inifile}
**.bgpConfig = xmldoc("BGPConfig.xml")
\end{inifile}

The configuration file may contain \ttt{<TimerParams>}, \ttt{<AS>} and
\ttt{Session} elements at the top level.

\begin{itemize}
  \item \ttt{<TimerParams>}: allows specifying various timing parameters
  for the routers.
  \item \ttt{<AS>}: defines Autonomous Systems, routers and rules to be applied.
  \item \ttt{<Session>}: specifies open sessions between edge routers. It must
  contain exactly two \ttt{<Router exterAddr="x.x.x.x"/>} elements.
\end{itemize}

\begin{verbatim}
<BGPConfig xmlns:xsi="http://www.w3.org/2001/XMLSchema-instance"
  xsi:schemaLocation="BGP.xsd">

  <TimerParams>
    <connectRetryTime> 120 </connectRetryTime>
    <holdTime> 180 </holdTime>
    <keepAliveTime> 60 </keepAliveTime>
    <startDelay> 15 </startDelay>
  </TimerParams>

  <AS id="60111">
    <Router interAddr="172.1.10.255"/> <!--Router A1-->
    <Router interAddr="172.1.20.255"/> <!--Router A2-->
  </AS>

  <AS id="60222">
    <Router interAddr="172.10.4.255"/> <!--Router B-->
  </AS>

  <AS id="60333">
    <Router interAddr="172.13.1.255"/> <!--Router C1-->
    <Router interAddr="172.13.2.255"/> <!--Router C2-->
    <Router interAddr="172.13.3.255"/> <!--Router C3-->
    <Router interAddr="172.13.4.255"/> <!--Router C4-->
    <DenyRouteOUT Address="172.10.8.0" Netmask="255.255.255.0"/>
    <DenyASOUT> 60111 </DenyASOUT>
  </AS>

  <Session id="1">
    <Router exterAddr="10.10.10.1" > </Router> <!--Router A1-->
    <Router exterAddr="10.10.10.2" > </Router> <!--Router C1-->
  </Session>

  <Session id="2">
    <Router exterAddr="10.10.20.1" > </Router> <!--Router A2-->
    <Router exterAddr="10.10.20.2" > </Router> <!--Router B-->
  </Session>

  <Session id="3">
    <Router exterAddr="10.10.30.1" > </Router> <!--Router B-->
    <Router exterAddr="10.10.30.2" > </Router> <!--Router C2-->
  </Session>
</BGPConfig>
\end{verbatim}

Inside \ttt{<AS>} elements various rules can be sepecified:
\begin{itemize}
  \item DenyRoute: deny route in IN and OUT traffic (\ttt{Address} and
        \ttt{Netmask} attributes must be specified.)
  \item DenyRouteIN : deny route in IN traffic (\ttt{Address} and
        \ttt{Netmask} attributes must be specified.)
  \item DenyRouteOUT: deny route in OUT traffic (\ttt{Address} and
        \ttt{Netmask} attributes must be specified.)
  \item DenyAS: deny routes learned by AS in IN  and OUT traffic (AS id must be
        specified as the body of the element.)
  \item DenyASIN : deny routes learned by AS in IN traffic (AS id must be
        specified as the body of the element.)
  \item DenyASOUT: deny routes learned by AS in OUT traffic (AS id must be
        specified as the body of the element.)
\end{itemize}

\nedtype{BgpRouter} is a \nedtype{Router} with the BGP protocol enabled.

%%% Local Variables:
%%% mode: latex
%%% TeX-master: "usman"
%%% End:


\cleardoublepage

\include{ch-diffserv}
\cleardoublepage

\chapter{The MPLS Models}
\label{cha:mpls}

\section{Overview}

Multi-Protocol Label Switching (MPLS) is a ``layer 2.5'' protocol for
high-performance telecommunications networks. MPLS directs data from one network
node to the next based on numeric labels instead of network addresses, avoiding
complex lookups in a routing table and allowing traffic engineering.
The labels identify virtual links (label-switched paths or LSPs, also
called MPLS tunnels) between distant nodes rather than endpoints. The routers
that make up a label-switched network are called label-switching routers (LSRs)
inside the network (``transit nodes''), and label edge routers (LER) on the
edges of the network (``ingress'' or ``egress'' nodes).

A fundamental MPLS concept is that two LSRs must agree on the meaning of the
labels used to forward traffic between and through them.
This common understanding is achieved by using signaling protocols by which one
LSR informs another of label bindings it has made. Such signaling protocols are
also called label distribution protocols. The two main label distribution
protocols used with MPLS are LDP and RSVP-TE.

INET provides basic support for building MPLS simulations. It provides models
for the MPLS, LDP and RSVP-TE protocols and their associated data structures,
and preassembled MPLS-capable router models. 

\section{Core Modules}

The core modules are:

\begin{itemize}
  \item \nedtype{Mpls} implements the MPLS protocol 
  \item \nedtype{LibTable} holds the LIB (Label Information Base)
  \item \nedtype{Ldp} implements the LDP signaling protocol for MPLS 
  \item \nedtype{RsvpTe} implements the RSVP-TE signaling protocol for MPLS 
  \item \nedtype{Ted} contains the Traffic Engineering Database 
  \item \nedtype{LinkStateRouting} is a simple link-state routing protocol
  \item \nedtype{SimpleClassifier} is a configurable ingress classifier for MPLS
\end{itemize}

\subsection{Mpls}

The \nedtype{Mpls} module implements the MPLS protocol. MPLS is situated between
layer 2 and 3, and its main function is to switch packets based on their labels.
For that, it relies on the data structure called LIB (Label Information Base).
LIB is fundamentally a table with the following columns: \textit{input-interface},
\textit{input-label}, \textit{output-interface}, \textit{label-operation(s)}.

Upon receiving a labelled packet from another LSR, MPLS first extracts the
incoming interface and incoming label pair, and then looks it up in local LIB. 
If a matching entry is found, it applies the prescribed label operations, and 
forwards the packet to the output interface. 

Label operations can be the following:

\begin{itemize}
  \item \textit{Push} adds a new MPLS label to a packet. (A packet may 
     contain multiple labels, acting as a stack.) When a normal IP packet
     enters an LSP, the new label will be the first label on the packet.
  \item \textit{Pop} removes the topmost MPLS label from a packet. 
     This is typically done at either the penultimate or the egress router.
  \item \textit{Swap}: Replaces the topmost label with a new label.
\end{itemize}

In INET, the local LIB is stored in a \nedtype{LibTable} module in the router.

Upon receiving an unlabelled (e.g. plain IPv4) packet, MPLS first determines the
forwarding equivalence class (FEC) for the packet using an ingress classifier, 
and then inserts one or more labels in the packet's newly created MPLS header. 
The packet is then passed on to the next hop router for the LSP.

The ingress classifier is also a separate module; it is selected depending 
on the choice of the signaling protocol.


\subsection{LibTable}

\nedtype{LibTable} stores the LIB (Label Information Base), as described
in the previous section. \nedtype{LibTable} is expected to have one instance
in the router. 

LIB is normally filled and maintained by label distribution protocols (RSVP-TE,
LDP), but in INET it is possible to preload it with initial contents.

The \nedtype{LibTable} module accepts an XML config file whose structure
follows the contents of the LIB table. An example configuration:

\begin{XML}
<libtable>
    <libentry>
        <inLabel>203</inLabel>
        <inInterface>ppp1</inInterface>
        <outInterface>ppp2</outInterface>
        <outLabel>
            <op code="pop"/>
            <op code="swap" value="200"/>
            <op code="push" value="300"/>
        </outLabel>
        <color>200</color>
    </libentry>
</libtable>
\end{XML}

There can be multiple \ttt{<libentry>} elements, each describing a row in the
table. Colums are given as child elements: \ttt{<inLabel>}, \ttt{<inInterface>},
etc. The \ttt{<color>} element is optional, and it only exists to be able to
color LSPs on the GUI. It is not used by the protocols.

\subsection{Ldp}

The \nedtype{Ldp} module implements the Label Distribution Protocol (LDP).
LDP is used to establish LSPs in an MPLS network when traffic engineering is not
required. It establishes LSPs that follow the existing IP routing table, and is
particularly well suited for establishing a full mesh of LSPs between all of the
routers on the network.

LDP relies on the underlying routing information provided by a routing protocol
in order to forward label packets. The router's forwarding information base, or
FIB, is responsible for determining the hop-by-hop path through the network.

In INET, the \nedtype{Ldp} module takes routing information from \nedtype{Ted}
module. The \nedtype{Ted} instance in the network is filled and maintained
by a \nedtype{LinkStateRouting} module. Unfortunately, it is currently not
possible to use other routing protocol implementations such as \nedtype{Ospf} 
in conjunction with \nedtype{Ldp}. 

When \nedtype{Ldp} is used as signaling protocol, it also serves as ingress
classifier for \nedtype{Mpls}.

\subsection{Ted}

The \nedtype{Ted} module contains the Traffic Engineering Database (TED). 
In INET, \nedtype{Ted} contains a link state database, including reservations
for each link by RSVP-TE.

\subsection{LinkStateRouting}

The \nedtype{LinkStateRouting} module provides a simple link state routing
protocol. It uses \nedtype{Ted} as its link state database. Unfortunately, the
\nedtype{LinkStateRouting} module cannot operate independently, it can only be
used inside an MPLS router.

 \subsection{RsvpTe}

The \nedtype{RsvpTe} module implements RSVP-TE (Resource Reservation Protocol --
Traffic Engineering), as signaling protocol for MPLS. RSVP-TE handles bandwidth
allocation and allows traffic engineering across an MPLS network. Like LDP, RSVP
uses discovery messages and advertisements to exchange LSP path information
between all hosts. However, whereas LDP is restricted to using the configured
IGP's shortest path as the transit path through the network, RSVP can take
taking into consideration network constraint parameters such as available
bandwidth and explicit hops. RSVP uses a combination of the Constrained Shortest
Path First (CSPF) algorithm and Explicit Route Objects (EROs) to determine how
traffic is routed through the network.

When \nedtype{RsvpTe} is used as signaling protocol, \nedtype{Mpls} needs a
separate ingress classifier module, which is usually a \nedtype{SimpleClassifier}.

The \nedtype{RsvpTe} module allows LSPs to be specified statically in an XML
config file. An example \ttt{traffic.xml} file:

TODO Figure out what stuff means. What is tunnel\_id, what is lspid? (which one is the label?)
which interface of host3 is used as endpoint?

\begin{XML}
<sessions>
    <session>
        <endpoint>host3</endpoint>
        <tunnel_id>1</tunnel_id>
        <paths>
            <path>
                <lspid>100</lspid>
                <bandwidth>100000</bandwidth>
                <route>
                    <node>10.1.1.1</node>
                    <lnode>10.1.2.1</lnode>
                    <node>10.1.4.1</node>
                    <node>10.1.5.1</node>
                </route>
                <permanent>true</permanent>
                <color>100</color>
            </path>
        </paths>
    </session>
</sessions>
\end{XML}

In the route, \ttt{<node>} stands for strict hop, and \ttt{<lnode>} for loose hop.

Paths can also be set up and torn down dynamically with \nedtype{ScenarioManager} 
commands (see chapter \ref{cha:scenario-scripting}). 
\nedtype{RsvpTe} understands the \ttt{<add-session>} and \ttt{<del-session>}
\nedtype{ScenarioManager} commands. The contents of the \ttt{<add-session>}
element can be the same as the \ttt{<session>} element for the \ttt{traffic.xml}
above. The \ttt{<del-command>} element syntax is also similar, but only
\ttt{<endpoint>}, \ttt{<tunnel\_id>} and \ttt{<lspid>} need to be specified.

The following is an example \ttt{scenario.xml} file:

\begin{XML}
<scenario>
    <at t="2">
        <add-session module="LSR1.rsvp">
            <endpoint>10.2.1.1</endpoint>
            <tunnel_id>1</tunnel_id>
            <paths>
                ...
            </paths>
        </add-session>
    </at>
    <at t="2.4">
        <del-session module="LSR1.rsvp">
            <endpoint>10.2.1.1</endpoint>
            <tunnel_id>1</tunnel_id>
            <paths>
                <path>
                    <lspid>100</lspid>
                </path>
            </paths>
        </del-session>
    </at>
</scenario>
\end{XML}

\section{Classifier}

The \nedtype{RsvpClassifier} module implements an ingress classifier for
\nedtype{Mpls} when using \nedtype{RsvpTe} for signaling. The classifier can be
configured with an XML config file.

\begin{inifile}
**.classifier.config = xmldoc("fectable.xml");
\end{inifile}

An example \ttt{fectable.xml} file:

\begin{XML}
<fectable>
    <fecentry>
        <id>1</id>
        <destination>host5</destination>
        <source>host1</source>
        <tunnel_id>1</tunnel_id>
        <lspid>100</lspid>
    </fecentry>
</fectable>
\end{XML}

TODO figure out what is id, tunnel\_id, lspid!

\section{MPLS-Enabled Router Models}

INET provides the following pre-assembled MPLS routers:

\begin{itemize}
  \item \nedtype{LdpMplsRouter} is an MPLS router with the LDP signaling protocol
  \item \nedtype{RsvpMplsRouter} is an MPLS router with the RSVP-TE signaling protocol
\end{itemize}

% HINT: A good MPLS primer:
% "MPLS for Dummies", Richard A Steenbergen <ras@nlayer.net>, nLayer Communications, Inc.
% https:www.nanog.org/meetings/nanog49/presentations/Sunday/mpls-nanog49.pdf


%%% Local Variables:
%%% mode: latex
%%% TeX-master: "usman"
%%% End:

\cleardoublepage

\chapter{Applications}
\label{cha:apps}


\section{Overview}

This chapter describes application models and traffic generators.
All applications implement the \nedtype{IApp} module interface
to ease configuring the \nedtype{StandardHost} module.

\section{TCP applications}

This sections describes the applications using the TCP protocol.
These applications use \msgtype{GenericAppMsg} objects to represent the data
sent between the client and server. The client message contains the expected
reply length, the processing delay, and a flag indicating that the connection
should be closed after sending the reply. This way intelligence (behaviour
specific to the modelled application, e.g. HTTP, SMB, database protocol) needs
only to be present in the client, and the server model can be kept simple and
dumb.


\subsection{TcpBasicClientApp}

Client for a generic request-response style protocol over TCP.
May be used as a rough model of HTTP or FTP users.

The model communicates with the server in sessions. During a session,
the client opens a single TCP connection to the server, sends several
requests (always waiting for the complete reply to arrive before
sending a new request), and closes the connection.

The server app should be \nedtype{TcpGenericServerApp}; the model sends
\msgtype{GenericAppMsg} messages.

Example settings:

FTP:

\begin{inifile}
numRequestsPerSession = exponential(3)
requestLength = truncnormal(20,5)
replyLength = exponential(1000000)
\end{inifile}

HTTP:

\begin{inifile}
numRequestsPerSession = 1 # HTTP 1.0
numRequestsPerSession = exponential(5)  # HTTP 1.1, with keepalive
requestLength = truncnormal(350,20)
replyLength = exponential(2000)
\end{inifile}

Note that since most web pages contain images and may contain frames,
applets etc, possibly from various servers, and browsers usually download
these items in parallel to the main HTML document, this module cannot
serve as a realistic web client.

Also, with HTTP 1.0 it is the server that closes the connection after
sending the response, while in this model it is the client.

\subsection{TcpSinkApp}

Accepts any number of incoming TCP connections, and discards whatever
arrives on them.

\subsection{TcpGenericServerApp}

Generic server application for modelling TCP-based request-reply style
protocols or applications.

The module accepts any number of incoming TCP connections, and expects
to receive messages of class \msgtype{GenericAppMsg} on them. A message should
contain how large the reply should be (number of bytes). \nedtype{TcpGenericServerApp}
will just change the length of the received message accordingly, and send
back the same message object. The reply can be delayed by a constant time
(\fpar{replyDelay} parameter).

\subsection{TcpEchoApp}

The \nedtype{TcpEchoApp} application accepts any number of incoming TCP
connections, and sends back the data that arrives on them, The byte counts are
multiplied by \fpar{echoFactor} before echoing. The reply can also be delayed by
a constant time (\fpar{echoDelay} parameter).

\subsection{TcpSessionApp}

Single-connection TCP application: it opens a connection, sends the given number
of bytes, and closes. Sending may be one-off, or may be controlled by a
``script'' which is a series of (time, number of bytes) pairs. May act either as
client or as server. Compatible with both IPv4 and IPv6.

\subsubsection*{Opening the connection}

Depending on the type of opening the connection (active/passive), the
application may be either a client or a server. In passive mode,
the application will listen on the given local local port, and wait for an
incoming connection. In active mode, the application will bind
to given local local address and local port, and connect to the
given address and port. It is possible to use an ephemeral port as
local port.

Even when in server mode (passive open), the application will only
serve one incoming connection. Further connect attempts will be
refused by TCP (it will send RST) for lack of LISTENing connections.

The time of opening the connection is in the \fpar{tOpen} parameter.

\subsubsection*{Sending data}

Regardless of the type of OPEN, the application can be made to send
data. One way of specifying sending is via the \fpar{tSend}, \fpar{sendBytes}
parameters, the other way is \fpar{sendScript}. With the former,
\fpar{sendBytes} bytes will be sent at \fpar{tSend}. When using 
\fpar{sendScript}, the format of the script is:

\begin{verbatim}
<time> <numBytes>; <time> <numBytes>;...
\end{verbatim}

\subsubsection*{Closing the connection}

The application will issue a TCP CLOSE at time \fpar{tClose}. If
\fpar{tClose=-1}, no CLOSE will be issued.



\subsection{TelnetApp}

Models Telnet sessions with a specific user behaviour.
The server app should be \nedtype{TcpGenericServerApp}.

In this model the client repeats the following activity
between \fpar{startTime} and \fpar{stopTime}:

\begin{enumerate}
\item Opens a telnet connection
\item Sends \fpar{numCommands} commands. The command is \fpar{commandLength} bytes long.
      The command is transmitted as entered by the user character by character, 
      there is \fpar{keyPressDelay} time between the characters. The server echoes
      each character. When the last character of the command is sent (new line),
      the server responds with a \fpar{commandOutputLength} bytes long message.
      The user waits \fpar{thinkTime} interval between the commands.
\item Closes the connection and waits \fpar{idleInterval} seconds
\item If the connection is broken, it is noticed after \fpar{reconnectInterval}
      and the connection is reopened
\end{enumerate}

Each parameter in the above description is ``volatile'', so you can
use distributions to emulate random behaviour.

\begin{note}
This module emulates a very specific user behaviour, and as such,
it should be viewed as an example rather than a generic Telnet model.
If you want to model realistic Telnet traffic, you are encouraged
to gather statistics from packet traces on a real network, and
write your model accordingly.
\end{note}

\subsection{TcpServerHostApp}

This module hosts TCP-based server applications. It dynamically creates
and launches a new ``thread'' object for each incoming connection.

Server threads can be implemented in C++. An example server thread class is
\cppclass{TcpGenericServerThread}.


\section{UDP applications}

The following UDP-based applications are implemented in INET:

\begin{itemize}
\item \nedtype{UdpBasicApp} sends UDP packets to a given IP address at a given interval
\item \nedtype{UdpBasicBurst} sends UDP packets to the given IP address(es) in bursts, or acts as a packet sink.
\item \nedtype{UdpEchoApp} is similar to \nedtype{UdpBasicApp}, but it sends back the packet after reception
\item \nedtype{UdpSink} consumes and prints packets received from the \nedtype{Udp} module
\item \nedtype{UdpVideoStreamClient},\nedtype{UdpVideoStreamServer} simulates video streaming over UDP
\end{itemize}

The next sections describe these applications in details.

\subsection{UdpBasicApp}

The \nedtype{UdpBasicApp} sends UDP packets to a the IP addresses given in the
\fpar{destAddresses} parameter. The application sends a message to one of the
targets in each \fpar{sendInterval} interval. The interval between message and
the message length can be given as a random variable. Before the packet is
sent, it is emitted in the \fsignal{sentPk} signal.

The application simply prints the received UDP datagrams. The \fsignal{rcvdPk}
signal can be used to detect the received packets.

\subsection{UdpSink}

This module binds an UDP socket to a given local port, and prints the
source and destination and the length of each received packet.

% TODO does not accept broadcast messages

\subsection{UdpEchoApp}

Similar to \nedtype{UdpBasicApp}, but it sends back the packet after reception.
It accepts only packets with \msgtype{UDPEchoAppMsg} type, i.e. packets that
are generated by another \nedtype{UdpEchoApp}.

When an echo response received, it emits an \fsignal{roundTripTime} signal.

\subsection{UdpVideoStreamClient}

This module is a video streaming client. It send one ``video streaming request'' to
the server at time \fpar{startTime} and receives stream from \nedtype{UdpVideoStreamServer}.

The received packets are emitted by the \fsignal{rcvdPk} signal.

\subsection{UdpVideoStreamServer}

This is the video stream server to be used with \nedtype{UdpVideoStreamClient}.

The server will wait for incoming "video streaming requests".
When a request arrives, it draws a random video stream size
using the \fpar{videoSize} parameter, and starts streaming to the client.
During streaming, it will send UDP packets of size \fpar{packetLen} at every
\fpar{sendInterval}, until \fpar{videoSize} is reached. The parameters \fpar{packetLen}
and \fpar{sendInterval} can be set to constant values to create CBR traffic,
or to random values (e.g. \ttt{sendInterval=uniform(1e-6, 1.01e-6)}) to
accomodate jitter.

The server can serve several clients, and several streams per client.

% FIXME why streamVector? VideoStreamData could be deleted immediately after last byte sent
% TODO this is video-on-demand, support multicast/broadcast video streaming too

\subsection{UdpBasicBurst}

Sends UDP packets to the given IP address(es) in bursts, or acts as a
packet sink. Compatible with both IPv4 and IPv6.

\subsubsection*{Addressing}

The \fpar{destAddresses} parameter can contain zero, one or more destination
addresses, separated by spaces. If there is no destination address given,
the module will act as packet sink. If there are more than one addresses,
one of them is randomly chosen, either for the whole simulation run,
or for each burst, or for each packet, depending on the value of the
\fpar{chooseDestAddrMode} parameter. The \fpar{destAddrRNG} parameter controls which
(local) RNG is used for randomized address selection.
The own addresses will be ignored.

An address may be given in the dotted decimal notation, or with the module
name. (The \cppclass{L3AddressResolver} class is used to resolve the address.)
You can use the "Broadcast" string as address for sending broadcast messages.

INET also defines several NED functions that can be useful:

\begin{itemize}
\item \ttt{moduleListByPath("pattern",...)}: \\
         Returns a space-separated list of the modulenames.
         All modules whose full path matches one of the pattern parameters will be included.
         The patterns may contain wilcards in the same syntax as in ini files.
         Example: 
\item \ttt{moduleListByNedType("fully.qualified.ned.type",...)}: \\
         Returns a space-separated list of the modulenames with the given NED type(s).
         All modules whose NED type name occurs in the parameter list will be included.
         The NED type name is fully qualified. Example: 
\end{itemize}

Examples:

\begin{inifile}
**.app[0].destAddresses = moduleListByPath("**.host[*]", "**.fixhost[*]")
**.app[1].destAddresses = moduleListByNedType("inet.nodes.inet.StandardHost")
\end{inifile}

The peer can be UDPSink or another UDPBasicBurst.

\subsubsection*{Bursts}

The first burst starts at \fpar{startTime}. Bursts start by immediately sending
a packet; subsequent packets are sent at \fpar{sendInterval} intervals. The
\fpar{sendInterval} parameter can be a random value, e.g. \ttt{exponential(10ms)}.
A constant interval with jitter can be specified as \ttt{1s+uniform(-0.01s,0.01s)}
or \ttt{uniform(0.99s,1.01s)}. The length of the burst is controlled by the
\fpar{burstDuration} parameter. (Note that if \fpar{sendInterval} is greater than
\fpar{burstDuration}, the burst will consist of one packet only.) The time between
burst is the \fpar{sleepDuration} parameter; this can be zero (zero is not
allowed for \fpar{sendInterval}.) The zero \fpar{burstDuration} is interpreted as infinity.

\subsubsection*{Operation as sink}

When \fpar{destAddresses} parameter is empty, the module receives packets and makes statistics only.


\section{SCTP applications}

TODO


\section{IPv4/IPv6 traffic generators}

The applications described in this section use the services of the network
layer only, they do not need transport layer protocols.
They can be used with both IPv4 and IPv6.

\nedtype{IIPvXTraffixGenerator} (prototype) sends IP or IPv6 datagrams to the
given address at the given \fpar{sendInterval}.
The \fpar{sendInterval} parameter can be a constant or a random value (e.g.
\ttt{exponential(1)}). If the \fpar{destAddresses} parameter contains more than
one address, one of them is randomly for each packet. An address may be given in
the dotted decimal notation (or, for IPv6, in the usual notation with colons),
or with the module name. (The \cppclass{L3AddressResolver} class is used to
resolve the address.) To disable the model, set \fpar{destAddresses} to "".

The \nedtype{IpvxTrafGen} sends messages with length \fpar{packetLength}.
The sent packet is emitted in the \fsignal{sentPk} signal.
The length of the sent packets can be recorded as scalars and vectors.

The \nedtype{IpvxTrafSink} can be used as a receiver of the packets
generated by the traffic generator. This module emits the packet
in the \fsignal{rcvdPacket} signal and drops it. The \ttt{rcvdPkBytes}
and \ttt{endToEndDelay} statistics are generated from this signal.

The \nedtype{IpvxTrafGen} can also be the peer of the traffic generators;
it handles the received packets exactly like \nedtype{IpvxTrafSink}.

\section{The PingApp application}

The \nedtype{PingApp} application
generates ping requests and calculates the packet loss and round trip
parameters of the replies.

Start/stop time, sendInterval etc. can be specified via parameters. An address
may be given in the dotted decimal notation (or, for IPv6, in the usual
notation with colons), or with the module name.
(The \cppclass{L3AddressResolver} class is used to resolve the address.)
To disable send, specify empty destAddr.

Every ping request is sent out with a sequence number, and replies are
expected to arrive in the same order. Whenever there's a jump in the
in the received ping responses' sequence number (e.g. 1, 2, 3, 5), then
the missing pings (number 4 in this example) is counted as lost.
Then if it still arrives later (that is, a reply with a sequence number
smaller than the largest one received so far) it will be counted as
out-of-sequence arrival, and at the same time the number of losses is
decremented. (It is assumed that the packet arrived was counted earlier as a loss,
which is true if there are no duplicate packets.)

Uses \msgtype{PingPayload} as payload for the ICMP(v6) Echo Request/Reply packets.


\section{Ethernet applications}

The \nedtype{inet.applications.ethernet} package contains modules
for a simple client-server application. The \nedtype{EtherAppClient} is a simple
traffic generator that peridically sends \msgtype{EtherAppReq} messages
whose length can be configured. destAddress, startTime,waitType, reqLength, respLength

The server component of the model (\nedtype{EtherAppServer}) responds with a
\msgtype{EtherAppResp} message of the requested length. If the response does
not fit into one ethernet frame, the client receives the data in multiple
chunks.

% FIXME reqLength>1500 causes an error in the LLC module
% FIXME numFrames field of EtherAppRes is not used
% FIXME server always sends 1497 byte chunks, it should depend on the framing (1497 is for LLC)
% FIXME if registerSAP is false (default), the and EtherLLC used, then the client won't receive messages (auto config?)
% FIXME Ieee802Nic -> EthernetInterface in the NED comment

Both applications have a \fpar{registerSAP} boolean parameter.
This parameter should be set to \ttt{true} if the application is connected
to the \nedtype{EtherLlc} module which requires registration of the SAP
before sending frames.

Both applications collects the following statistics: sentPkBytes, rcvdPkBytes,
endToEndDelay.

The client and server application works with any model that accepts
Ieee802Ctrl control info on the packets (e.g. the 802.11 model).
The applications should be connected directly to the \nedtype{EtherLlc}
or an EthernetInterface NIC module.

The model also contains a host component that groups the applications
and the LLC and MAC components together (\nedtype{EtherHost}). This node does
not contain higher layer protocols, it generates Ethernet traffic directly.
By default it is configured to use half duplex MAC (CSMA/CD).



%%% Local Variables:
%%% mode: latex
%%% TeX-master: "usman"
%%% End:


\cleardoublepage

\include{ch-authors-guide}
\cleardoublepage

\bibliographystyle{alpha}
\bibliography{inet-developers-guide}


%% no need for the following since 'tocbibind' package
%% \phantomsection
%% \addcontentsline{toc}{chapter}{\indexname}
\printindex

\end{document}
